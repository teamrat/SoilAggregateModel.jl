%\documentclass[SOIL]{copernicus}
\documentclass[SOIL, manuscript]{copernicus}

%% \usepackage commands included in the copernicus.cls:
%\usepackage[german, english]{babel}
%\usepackage{tabularx}
%\usepackage{cancel}
%\usepackage{multirow}
%\usepackage{supertabular}
%\usepackage{algorithmic}
%\usepackage{algorithm}
%\usepackage{amsthm}
%\usepackage{float}
%\usepackage{subfig}
%\usepackage{rotating}

% Title and authors

\begin{document}

\title{Hierarchical Soil Aggregate Formation and Carbon Fate: A Multi-Scale Model of Biogeochemical Hotspots}


% \Author[affil]{given_name}{surname}

\Author[1]{Teamrat A.}{Ghezzehei}
\Author[2]{Dani}{Or}

\affil[1]{University of California, Merced}
\affil[2]{University of Nevada, Reno}


\correspondence{Teamrat A. Ghezzehei (taghezzehei@ucmerced.edu)}

\runningtitle{Life Cyle of Soil Aggregates}

\runningauthor{Ghezzehei and Or}





\received{}
\pubdiscuss{} %% only important for two-stage journals
\revised{}
\accepted{}
\published{}

%% These dates will be inserted by Copernicus Publications during the typesetting process.


\firstpage{1}

\maketitle



\begin{abstract}
TEXT
\end{abstract}


\copyrightstatement{TEXT} %% This section is optional and can be used for copyright transfers.


\introduction  
Soil aggregates are characterized by sieving—a destructive method that removes the spatial organizational context aggregates occupy within intact soil structure \citep{Ghezzehei2024}. This measurement approach conflates two fundamentally different phenomena: mechanical fragments formed instantly by tillage or freeze-thaw cycles, and hierarchical aggregates resulting from slow biological binding processes—multi-scale structures formed through microbial and root activity around organic matter cores \citep{Tisdall1982,Ghezzehei2024,Or2021}. Although mechanical and hierarchical aggregates are indistinguishable after separation, their genesis, stability mechanisms, and ecological functions differ profoundly. This ambiguity in how we define and measure soil aggregates conceals critical differences in aggregate formation pathways and obscures cause-effect relationships between aggregation and soil functions \citep{Ghezzehei2024}.

Despite decades of research linking aggregation to soil carbon storage, aeration, and structural stability, mechanistic models that predict the degree of aggregation from soil properties and management practices remain undeveloped. Conversely, no mechanistic frameworks exist that predict physical and biogeochemical functions from measured aggregation state. While empirical correlations between aggregate stability indices and soil properties are widely reported, these relationships break down across soil types, climates, and management histories, limiting their utility for forecasting responses to land-use change or designing management interventions. Statistical and machine learning approaches provide pattern recognition within calibration ranges, yet they cannot answer mechanistic questions about why aggregation responds to particular drivers or how aggregate properties control carbon cycling processes.

This predictive void has different origins for the two aggregate formation pathways. Mechanical aggregates formed by tillage have some modeling frameworks \citep{Ghezzehei2000,Ghezzehei2003}, but hierarchical biological aggregates lack mechanistic representation. Despite their dominance in undisturbed soils and central role in carbon cycling models, frameworks linking particulate organic matter inputs to aggregate formation dynamics remain absent. Recent pore-scale modeling \citep{Prechtel2023} addresses mucilage deposition but not full aggregate life cycles or population-level dynamics. Fundamental questions about spatial organization around POM cores, temporal evolution, and emergent soil properties remain unresolved. The remainder of this paper focuses on hierarchical biological aggregation.

The attribution of soil functions to hierarchical aggregates rests on assumptions about their physical organization that may not reflect their structure in intact soils. Claims that inter-aggregate pores provide preferential pathways for aeration and infiltration \citep{Horn2005,Smucker2007}, that anoxic cores within aggregates protect soil organic carbon from decomposition \citep{Keiluweit2016}, and that physical occlusion within aggregates limits microbial access to substrates \citep{Kravchenko2015,Plante2009} all require aggregates to exist as discrete, bounded entities with differentiated inter- versus intra-aggregate spaces. However, conceptualization of aggregates as detached, isolated spheres within soil is physically inconsistent with how aggregates are embedded in intact soil structure \citep{Baveye2024}. Attempts to delineate hierarchical aggregates in X-ray computed tomography images of intact soil have proven challenging \citep{Koestel2021}, and the soil science community remains divided between aggregate-centric and pore-centric perspectives for quantifying soil functions \citep{Kravchenko2024}. While micro-scale gradients may exist below imaging resolution, the absence of clear boundaries at millimeter scales contrasts with the discrete-entity conceptualization underlying functional interpretations. Large inter-aggregate pores characteristic of discrete entities are absent except following mechanical disturbance. Yet aggregation metrics consistently correlate with these soil functions. This creates a fundamental tension: Are aggregates causing these functions, or are both aggregation and the attributed functions emerging from common underlying drivers such as root activity, biopore formation, and exudate deposition \citep{Or2021}? These tensions point toward a fundamental reinterpretation of aggregate genesis.

Mounting evidence suggests an alternative interpretation. Rather than aggregates forming first and capturing particulate organic matter, POM deposition may initiate aggregate formation through localized biogeochemical processes that produce binding agents and stabilized zones. Section~\ref{modeling framework} develops this POM-centric hypothesis and presents the modeling framework needed to test it mechanistically. \textbf{We treat hierarchical aggregates as consequences of POM-driven biogeochemical processes, not pre-existing structures that protect organic matter.}

Despite widespread invocation of aggregate-scale processes in conceptual models of soil carbon cycling, no existing mechanistic models represent this spatial scale for hierarchical aggregates. A comprehensive review of 71 microbial soil organic carbon models found that all operate as zero-dimensional bulk soil representations or one-dimensional vertical profiles \citep{Chandel2023}. None simultaneously incorporate aggregate-scale spatial resolution while representing gradients in carbon, oxygen, microbial communities, or mineral-associated organic carbon within individual aggregates. Without aggregate-scale resolution, quantitative predictions about processes most frequently invoked to explain aggregate function---anoxic core formation, physical occlusion, and diffusion limitation---remain inaccessible within existing soil carbon modeling frameworks. These mechanisms have been assumed and debated for decades but not yet tested using models that link aggregate-scale gradients, life cycles, and population dynamics to soil-scale carbon pools.

Here we develop a multi-scale mathematical framework that represents individual aggregates as POM-centered biogeochemical hotspots and scales heterogeneous aggregate populations to emergent soil-level properties. The framework is evaluated against laboratory incubation and long-term field chronosequence data at the macroscopic scale. Beyond this evaluation, the model generates a set of specific, falsifiable predictions about aggregate internal dynamics, population-scale behavior, and carbon fate that are now amenable to targeted experimental testing. We present these predictions explicitly as a contribution independent of the model itself.

\section{Modeling Framework}
\label{modeling framework}

\subsection{The POM-Centric Hypothesis}

Multiple lines of evidence demonstrate that particulate organic matter occupies the centers of hierarchical aggregates rather than being captured by pre-existing structures. Scanning electron microscopy reveals POM cores in stable microaggregates, with many retaining the elongated morphology of root fragments \citep{Tisdall1982}. High-resolution X-ray computed tomography shows root-derived organic matter at the centers of stable macroaggregates \citep{Kravchenko2015}. Isotopic labeling demonstrates that aggregate formation occurs concurrently on fresh litter surfaces, with organic matter occlusion and organo-mineral associations developing around the litter nucleus \citep{Witzgall2021}. The degree of aggregation correlates strongly with root density \citep{Cockroft2000}, and fresh POM inputs act as binding nuclei, triggering macroaggregate formation on timescales of weeks \citep{DeGryze2006,Pronk2012,Poeplau2025}. POM-associated carbon constitutes the dominant fraction in sieved aggregates across diverse soil types \citep{Jastrow1996,Gale2000}.

Rather than aggregates forming first and subsequently capturing particulate organic matter, this evidence supports a reversed sequence: POM deposition initiates localized microbial activity, binding agents accumulate around the resource hotspot through extracellular polymeric substance production and hyphal growth, and a stabilized zone develops. Sieving later reveals this stabilized zone as an "aggregate." This interpretation redefines aggregates as consequences of POM-driven biogeochemical processes rather than pre-existing protective structures.

\subsection{Common-Driver Mechanism and Modeling Implications}

This POM-centric mechanism provides a concrete realization of the common-driver interpretation proposed by \citet{Or2021}. Root activity simultaneously deposits POM cores that nucleate aggregate formation, creates macroporosity through mechanical penetration and turnover, alters water dynamics through mucilage exudation and water uptake, and fuels microbial binding through substrate provision. Aggregation and the attributed functions—enhanced carbon storage, preferential flow pathways, oxygen gradients—thus co-emerge from shared root-driven processes rather than aggregates causing these functions through physical protection or structural isolation.

This framework explains several patterns that challenge the discrete-entity interpretation. It accounts for why POM consistently occupies aggregate cores across soil types and management histories. It explains why aggregate formation follows POM inputs temporally, with macroaggregates appearing within weeks of fresh organic matter addition. It clarifies why aggregation responds dynamically to changes in organic matter inputs and disturbance regimes, tracking the availability of nucleating substrates rather than reflecting stable physical structures. These observations do not diminish the reality or importance of aggregates as zones of enhanced stability and biogeochemical activity. Rather, they suggest redefining aggregates as POM-centered biogeochemical hotspots that evolve through time as resources are consumed and binding agents degrade.

Representing this conceptual framework mechanistically requires addressing multiple gaps in current modeling approaches. Spatial representations of how dissolved carbon, oxygen, and microbial communities vary with distance from POM cores have not been developed, despite widespread invocation of gradients in conceptual models. Temporal frameworks for tracking aggregate life cycles from POM input through binding agent production, peak stability, resource depletion, and eventual disaggregation remain absent. Population dynamics approaches that scale heterogeneous individual aggregates to emergent soil-level properties are not available, yet soil-scale carbon cycling necessarily reflects the collective behavior of aggregate populations with diverse sizes, ages, and developmental stages. Such population frameworks, analogous to population balance models in chemical and biological systems \citep{Ramkrishna2014}, would treat aggregates as entities with internal coordinates—age since formation, current size, and POM depletion state—that evolve through nucleation around fresh inputs and eventual resource exhaustion.

Without mechanistic representations that capture these spatial, temporal, and population-scale dimensions, critical questions remain unanswerable. We cannot forecast how aggregation will respond to management interventions or environmental change, limiting our ability to design practices that promote beneficial aggregation or mitigate degradation. Protection mechanisms invoked in conceptual models—anoxic core formation, physical occlusion of substrates, diffusion limitation of oxygen—remain untested, creating extrapolation risk when empirical relationships calibrated at one site are applied elsewhere. We cannot quantitatively test competing hypotheses about aggregate formation pathways or resolve whether discrete-entity or POM-centric interpretations better explain observed patterns. Mechanistic modeling that explicitly represents aggregate-scale processes would enable prediction of aggregate formation rates and size distributions from POM input characteristics, testing of protection mechanism hypotheses through simulation of oxygen and carbon gradients, quantification of how aggregate populations determine emergent soil-scale carbon residence times, and explicit linkage of aggregate-scale biogeochemical processes to mineral-associated organic carbon formation and long-term carbon storage.

\subsection{Multi-Scale Modeling Strategy}

Our modeling framework addresses these gaps through a multi-scale approach that couples aggregate-scale biogeochemistry with population-scale emergence. Individual aggregates are represented as spherical biogeochemical domains with particulate organic matter cores at the center, while soil volumes are represented as populations of heterogeneous aggregate cohorts with diverse sizes, ages, and developmental stages.

\subsubsection{Individual Aggregate Scale}

Each biogeochemical domain is a spherical volume centered on a particulate organic matter deposit, with radial coordinate $r$ extending from $r = 0$ to $r_{\max}$. The POM core occupies $0 \leq r \leq r_0$ and the surrounding soil matrix occupies $r_0 \leq r \leq r_{\max}$. Binding agent concentrations develop radial gradients that determine where cohesive strength exceeds disaggregation stress; a stable aggregate emerges only where this criterion is met and becomes apparent upon sieving (Section~2.4.7).

The POM core is a structurally rigid, well-mixed solid reservoir. Internal diffusion is non-limiting relative to surface dissolution, so $P(t)$ is spatially uniform. As dissolution proceeds, $P$ decreases but the core geometry remains unchanged. This is consistent with X-ray computed tomography observations of void formation around partially decomposed particulate organic matter \citep{REFERENCE}. Dissolution flux enters $C$ as a boundary condition at $r = r_0$ (Section~2.4.6).

The nine state variables $\mathbf{u} = [P, C, M, B, F_i, F_n, F_m, E, O]^T$ evolve according to:
%
\begin{equation}
\frac{\partial \mathbf{u}}{\partial t} = r^{-2} \frac{\partial}{\partial r}\left(r^2 \mathbf{D} \frac{\partial \mathbf{u}}{\partial r}\right) + \mathbf{S}(\mathbf{u}, r, t)
\end{equation}
%
where $\mathbf{D}$ is a diagonal matrix of effective diffusion coefficients and $\mathbf{S}$ contains biogeochemical source and sink terms. The variables represent particulate organic matter ($P$), soluble organic carbon ($C$, including dissolved and equilibrium-sorbed phases), mineral-associated organic carbon ($M$), bacterial biomass ($B$), three fungal pools ($F_i$, $F_n$, $F_m$), extracellular polymeric substances ($E$), and oxygen ($O$). The three fungal compartments distinguish insulated immobile fungi forming mature hyphal structures ($F_i$), non-insulated immobile fungi representing active hyphal tips ($F_n$), and mobile fungi representing internal resource transport ($F_m$). 


All domains share the same environmental forcing: soil temperature $T(t)$, matric potential $\psi(t)$, and ambient oxygen concentration $O_{\text{amb}}(t)$. Temperature governs microbial rate coefficients through Arrhenius kinetics with process-specific activation energies, diffusion coefficients through molecular transport theory, and gas--liquid partitioning through thermodynamic equilibrium (Section~\ref{sec:temperature}).Matric potential determines volumetric water content $\theta$ locally through a van Genuchten retention curve whose parameters are modified by EPS and fungal hyphae (Section~2.4.8); air-filled porosity follows as $\theta_a = \theta_s - \theta$. Ambient oxygen sets the outer boundary condition for $O$.

\subsubsection{Population Scale}

Soil volumes are represented as ensembles of independent biogeochemical domains with diverse sizes and ages. Particulate organic matter inputs occur at discrete intervals with prescribed size distributions, creating new cohorts representing different POM size classes. Each cohort $(i,j)$—indexed by input event $i$ and size class $j$—evolves independently under the shared environmental forcing according to Eq.~(\ref{eq:master}). This independence is a mean-field approximation: no diffusive coupling between domains is represented, and cohorts occupy non-overlapping representative volume elements within the soil.

The spatially integrated state of cohort $(i,j)$ over domain $[0, r^*]$ is:
%
\begin{equation}
\mathbf{U}_{i,j}(r^*; t) = 4\pi \int_0^{r^*} \mathbf{u}_{i,j}(r,t)\, r^2\, dr
\end{equation}
%
The choice of upper limit $r^*$ determines the physical meaning: $r^* = r_{\max}$ yields total carbon associated with the cohort, while $r^* = r_{\text{agg}}(t)$ yields carbon within the stable zone. The population state at time $t$ is obtained by summing over all cohorts:
%
\begin{equation}
\mathbf{U}_{\text{soil}}(t) = \sum_{i=1}^{I(t)} \sum_{j=1}^{N_{\text{size}}} f_{i,j} \cdot \mathbf{U}_{i,j}(r^*; t)
\end{equation}
%
where $I(t)$ is the number of input events by time $t$ and $f_{i,j}$ is the number frequency of cohort $(i,j)$. Young cohorts are dominated by active POM decomposition and microbial dynamics, while old cohorts are dominated by persistent mineral-associated organic carbon and residual binding agents. Emergent soil-scale properties—bulk carbon pools, aggregate size distributions, age structure, and carbon residence times—arise from integrating across the heterogeneous population and are detailed in subsequent subsections.

\subsubsection{Size Distribution and Frequency Conversion}

POM size distributions from experimental measurements (e.g., sieving, image analysis) are typically characterized as mass-based frequencies. Population modeling requires number-based frequencies that determine cohort abundances:
%
\begin{equation}
f_{n}(d_0) = \frac{f_{m}(d_0)}{\rho_{\text{POM}} \cdot \frac{\pi d_0^3}{6}}
\end{equation}
%
where $f_m(d_0)$ and $f_n(d_0)$ are mass- and number-based frequency densities for POM diameter $d_0$, and $\rho_{\text{POM}}$ is the POM density. The number frequency of cohort $(i,j)$ entering at input event $i$ in size class $j$ is:
%
\begin{equation}
f_{i,j} = f_{n}(d_{0,j}) \cdot \Delta d_j \cdot V_{\text{REV}} \cdot \phi_{\text{POM},i}
\end{equation}
%
where $\Delta d_j$ is the size class width, $V_{\text{REV}}$ is the representative elementary volume, and $\phi_{\text{POM},i}$ is the volumetric POM fraction at input event $i$. The representative elementary volume is defined such that the least frequent size class has an expected count of one:
%
\begin{equation}
V_{\text{REV}} = \frac{1}{\min_j\left[f_{n}(d_{0,j}) \cdot \Delta d_j \cdot \phi_{\text{POM}}\right]}
\end{equation}
%

\subsubsection{Emergent Soil-Scale Properties}

Soil-scale properties arise from integrating individual domain states across the heterogeneous population. The choice of integration domain $r^*$ in $\mathbf{U}_{i,j}(r^*; t)$ distinguishes two physically meaningful carbon pools: $r^* = r_{\max}$ yields total carbon associated with each cohort in intact soil, while $r^* = r_{\text{agg}}(t)$ yields carbon that would survive wet sieving. Both are obtained from the same radial profiles without additional computation.

\textbf{Bulk carbon pools.} The integrated amount of any carbon pool within cohort $(i,j)$ is obtained by extracting the corresponding component of $\mathbf{U}_{i,j}(r^*; t)$. For example, the total MAOC at soil scale is:
%
\begin{equation}
U_{\text{soil}}^M(t) = \sum_{i=1}^{I(t)} \sum_{j=1}^{N_{\text{size}}} f_{i,j} \cdot U_{i,j}^M(r^*; t)
\end{equation}
%
and analogously for $P$, $C$, $B$, $F_i$, $F_n$, $F_m$, and $E$. The superscript denotes the pool; the choice of $r^*$ selects bulk or aggregate-associated carbon as above.

\textbf{Aggregate size distribution.} The population size distribution evolves as binding agents accumulate and degrade:
%
\begin{equation}
n(d,t) = \sum_{i=1}^{I(t)} \sum_{j=1}^{N_{\text{size}}} f_{i,j}\, \delta\!\left(d - d_{\text{agg}}^{(i,j)}(t)\right)
\end{equation}
%
where $d_{\text{agg}}^{(i,j)}(t)$ is the stable aggregate diameter of cohort $(i,j)$ determined by the stability criterion and $\delta$ is the Dirac delta function. The distribution $n(d,t)$ is a discrete, measure-valued sum over cohorts; any smooth size distribution presented in Results is a kernel-smoothed visualization.

\textbf{Age structure.} The population age structure in size-age space is:
%
\begin{equation}
\mathcal{A}(a,d,t) = \sum_{\substack{i:\, t - t_i = a}} \;\sum_{\substack{j:\, d_{\text{agg}}^{(i,j)}(t) \in [d - \Delta d/2,\, d + \Delta d/2]}} f_{i,j}
\end{equation}
%
where $t_i$ is the time of input event $i$ and $a = t - t_i$ is cohort age. This quantifies system memory: current soil properties depend on the full history of POM inputs and environmental conditions. The mean age in size class $d$ is:
%
\begin{equation}
\bar{a}(d,t) = \frac{\sum_a a\, \mathcal{A}(a,d,t)}{\sum_a \mathcal{A}(a,d,t)}
\end{equation}
%

\textbf{Carbon residence time.} The fraction of total carbon from cohort $(i,j)$ remaining at time $t$ is:
%
\begin{equation}
P_{\text{surv}}^{(i,j)}(t) = \frac{\mathbf{1}^T \mathbf{U}_{i,j}(r_{\max}; t)}{\mathbf{1}^T \mathbf{U}_{i,j}(r_{\max}; 0)}
\end{equation}
%
where $\mathbf{1}^T \mathbf{U}_{i,j}$ sums all carbon-bearing components of the integrated state vector. Residence time uses the full-domain integral ($r_{\max}$): carbon beyond the stable zone remains in the soil matrix even if it would not survive sieving.

\subsection{Individual Aggregate Biogeochemistry}

\subsubsection{POM Dissolution}

The primary carbon source within each domain is dissolution of the POM core. The surface flux $R_P$ is governed by enzymatic and passive dissolution processes that depend on water availability, oxygen supply, and microbial activity at the POM surface \citep{Ebrahimi2015}. POM depolymerization requires extracellular enzymes produced by bacteria and fungi colonizing the POM surface; we make this dependence explicit through dual Monod terms for microbial biomass alongside the abiotic limitations. The POM depletion rate [$\mu$g-C\,day$^{-1}$], which is also the rate of carbon addition at the inner boundary, is:
%
\begin{equation}
\begin{split}
R_P =\;& 4\pi r_0^2 \, R_P^{\max}(T) \, \frac{P}{P_0} \, \frac{B_0}{K_{B,P} + B_0} \, \frac{F_{n,0}}{K_{F,P} + F_{n,0}} \\
& \times \frac{\theta_0}{\theta_P + \theta_0} \, \frac{O_{\text{aq},0}}{L_P + O_{\text{aq},0}}
\end{split}
\end{equation}
%
where the subscript $0$ denotes evaluation at the POM surface ($r = r_0$), $R_P^{\max}(T)$ is the maximum dissolution rate scaling with temperature via $\mathcal{E}_{a,P}$ (Section~\ref{sec:temperature}), $K_{B,P}$ and $K_{F,P}$ are half-saturation concentrations for bacterial and non-insulated fungal contributions to enzymatic dissolution, $\theta_P$ is the half-saturation water content, and $L_P$ is the half-saturation dissolved oxygen concentration. The initial POM radius $r_0$ and initial POM mass $P_0$ are fixed; the factor $P/P_0$ reduces dissolution as the POM core is consumed, ensuring $P$ decays monotonically to zero. The corresponding flux density at the POM surface is:
%
\begin{equation}
J_P = \frac{R_P}{4\pi r_0^2} = R_P^{\max}(T) \, \frac{P}{P_0} \, \frac{B_0}{K_{B,P} + B_0} \, \frac{F_{n,0}}{K_{F,P} + F_{n,0}} \, \frac{\theta_0}{\theta_P + \theta_0} \, \frac{O_{\text{aq},0}}{L_P + O_{\text{aq},0}}
\end{equation}
%
where $J_P$ [$\mu$g-C\,mm$^{-2}$\,day$^{-1}$] is the mass flux per unit area of POM surface, and $R_P^{\max}(T)$ has the same units.

\subsubsection{MAOC Formation Dynamics}

MAOC formation is represented as a \textbf{two-stage process} that explicitly separates fast partitioning from slow stabilization. Stage 1 describes rapid concentration-driven sorption $\leftrightarrow$ desorption at mineral surfaces (reversible), while Stage 2 captures rate-limited structural incorporation through micropore diffusion, ligand exchange, and organo-mineral complexation (hysteretic).

\textbf{Stage 1: Fast equilibrium sorbed pool.} Dissolved organic carbon ($C_{\text{aq}}$) exchanges rapidly with a weakly sorbed interfacial pool ($C_{\text{eq}}$) at near-instantaneous equilibrium:
%
\begin{equation}
C_{\text{aq}} = \frac{C}{\theta + \rho_b k_d}, \quad C_{\text{eq}} = k_d \, C_{\text{aq}}
\end{equation}
%
where $k_d$ is the linear partition coefficient. $C_{\text{eq}}$ is not an independent state variable: it is an algebraic partition of $C$, so Stage 1 does not alter total carbon accounting. During drying, $\theta$ decreases while $C$ increases, causing $C_{\text{eq}}$ to rise and provide elevated substrate for Stage 2.

\textbf{Stage 2: Slow MAOC stabilization.} Association into MAOC ($M$) occurs through rate-limited transformation from $C_{\text{eq}}$, governed by asymmetric first-order kinetics toward local equilibrium:
%
\begin{equation}
\frac{\partial M}{\partial t} = \kappa_s(T)\;\phi_\varepsilon(M_{\text{eq}} - M) \;-\; \kappa_d(T)\;\phi_\varepsilon(M - M_{\text{eq}})
\label{eq:maoc_smooth}
\end{equation}
%
where the softplus function
%
\begin{equation}
\phi_\varepsilon(x) = \varepsilon\,\ln\!\left(1 + e^{x/\varepsilon}\right)
\label{eq:softplus}
\end{equation}
%
provides a $C^\infty$ approximation to $\max(0,x)$ with smoothing width $\varepsilon = 0.01\;\mu\text{g\,mm}^{-3}$, negligible relative to typical MAOC concentrations ($M \sim 1$--$10\;\mu\text{g\,mm}^{-3}$). The regularization ensures continuous derivatives everywhere, which is important for implicit time integration requiring Jacobian evaluation near the switching surface.

The rate constants $\kappa_s(T)$ and $\kappa_d(T)$ scale with temperature via Arrhenius kinetics with distinct activation energies $\mathcal{E}_{a,s}$ and $\mathcal{E}_{a,d}$ (Section~\ref{sec:temperature}), and satisfy $\kappa_d < \kappa_s$ at the reference temperature, representing the kinetic asymmetry that favors carbon accumulation over release. The asymmetric rates generate path-dependent hysteresis: stabilization occurs faster than release, causing MAOC to remain elevated after wetting-drying cycles. Critically, this switching structure enables true hysteresis---unlike competing-process models where both sorption and desorption remain active simultaneously, a mathematical structure that cannot produce hysteresis loops under oscillating environmental conditions.

The kinetic asymmetry is itself temperature-dependent. The hysteresis ratio
%
\begin{equation}
\frac{\kappa_s(T)}{\kappa_d(T)} = \frac{\kappa_{s,\text{ref}}}{\kappa_{d,\text{ref}}} \exp\!\left[\frac{\mathcal{E}_{a,s} - \mathcal{E}_{a,d}}{R}\left(\frac{1}{T_{\text{ref}}} - \frac{1}{T}\right)\right]
\label{eq:hysteresis_ratio}
\end{equation}
%
decreases with warming when $\mathcal{E}_{a,d} > \mathcal{E}_{a,s}$, as expected for organo-mineral bond disruption (desorption) versus diffusion-limited surface adsorption (sorption). This predicts that wetting--drying cycles generate less MAOC hysteresis in warmer soils---a mechanism of MAOC destabilization distinct from, and additive with, enhanced microbial mineralization.

The equilibrium MAOC concentration follows a Langmuir-Freundlich isotherm that accounts for mineral surface heterogeneity:
%
\begin{equation}
M_{\text{eq}} = \frac{M_{\max}\,(k_L C_{\text{eq}})^{n_{\text{LF}}}}{1 + (k_L C_{\text{eq}})^{n_{\text{LF}}}}
\end{equation}
%
where $M_{\max}$ is the maximum sorption capacity, $k_L$ is the Langmuir affinity constant, and $n_{\text{LF}}$ is the Freundlich heterogeneity parameter ($n_{\text{LF}} < 1$ for heterogeneous sites; $n_{\text{LF}} = 1$ recovers standard Langmuir). Equilibrium is defined in terms of $C_{\text{eq}}$, not $C$ directly, which is the origin of the two-stage moisture coupling mechanism.

The maximum MAOC capacity depends on soil mineralogy and texture:
%
\begin{equation}
M_{\max} = k_{\text{ma}}\, f_{\text{clay+silt}}\, \rho_b
\end{equation}
%
where $k_{\text{ma}}$ is the mineral activity coefficient and $f_{\text{clay+silt}}$ is the mass fraction of clay and silt particles providing reactive surface area.

\subsubsection{Microbial Uptake and Allocation}
\textbf{Uptake kinetics.} Microbial carbon uptake follows dual Monod kinetics representing limitation by both substrate availability and oxygen, with sensitivity to water potential:
%
\begin{align}
R_B &= r_{B,\max}(T) \frac{C_{\text{aq}}}{K_B + C_{\text{aq}}} \frac{O_{\text{aq}}}{L_B + O_{\text{aq}}} B \, e^{\nu_B \psi} \\
R_F &= r_{F,\max}(T) \frac{C_{\text{aq}}}{K_F + C_{\text{aq}}} \frac{O_{\text{aq}}}{L_F + O_{\text{aq}}} (F_i + \lambda F_n) \, e^{\nu_F \psi}
\end{align}
%
where $r_{B,\max}(T)$ and $r_{F,\max}(T)$ are maximum specific uptake rates scaling with temperature via $\mathcal{E}_{a,B}$ and $\mathcal{E}_{a,F}$ respectively (Section~\ref{sec:temperature}), $K_B$, $K_F$ and $L_B$, $L_F$ are half-saturation constants for carbon and oxygen respectively, and $\nu_B$, $\nu_F$ are water potential sensitivity coefficients with $\nu_F < \nu_B$ reflecting greater drought tolerance of fungi. Only $F_i$ and $F_n$ contribute to fungal uptake; $\lambda \ll 1$ reflects the small fraction of non-insulated hyphae at active uptake surfaces. The aqueous oxygen concentration is:
%
\begin{equation}
O_{\text{aq}} = \frac{O \theta}{\theta + K_H(T) \theta_a}
\end{equation}
%
where $K_H(T)$ is the temperature-dependent dimensionless Henry's law constant (Section~\ref{sec:temperature}), and $C_{\text{aq}}$ is defined in Section~2.4.2 (Stage 1).

\textbf{Bacterial allocation.} Basal maintenance metabolism requires a minimum carbon flux:
%
\begin{equation}
R_{B,b} = r_{B,\max}(T) \frac{C_B}{K_B + C_B} \frac{O_{\text{aq}}}{L_B + O_{\text{aq}}} B \, h_B
\end{equation}
%
where $C_B$ is the basal carbon requirement and $h_B$ is a sigmoid that smoothly reduces maintenance as biomass approaches the minimum viable concentration $B_{\min}$:
%
\begin{equation}
h_B = \frac{\exp(\beta B)}{\exp(\beta B) + \exp(\beta B_{\min})}
\end{equation}
%
with $\beta = 50/B_{\min}$. This ensures $R_{B,b}$ vanishes faster than $R_B$ as $B \to 0$, guaranteeing non-negative biomass without an explicit extinction threshold. The excess uptake $R_{\text{diff}} = R_B - R_{B,b}$ determines the allocation regime. When positive, a fraction is assimilated at yield:
%
\begin{equation}
Y_B = \frac{Y_{B,\max} \max(0,\, R_{\text{diff}})}{\max(0,\, R_{\text{diff}}) + K_Y}
\end{equation}
%
where $Y_{B,\max}$ is the maximum yield and $K_Y$ is a half-saturation constant. Assimilated carbon is partitioned between biomass growth and EPS production:
%
\begin{align}
\Gamma_B &= Y_B \max(0,\, R_{\text{diff}})(1-\gamma) + \min(0,\, R_{\text{diff}}) \\
\Gamma_E &= Y_B \max(0,\, R_{\text{diff}})\, \gamma
\end{align}
%
where $\gamma$ is the EPS allocation fraction. When $R_{\text{diff}} < 0$, $Y_B = 0$ and $\Gamma_B = R_{\text{diff}} < 0$: the deficit is met by biomass catabolism with no growth or EPS production. Total respiration is:
%
\begin{equation}
\text{Resp}_B = R_{B,b} + \max(0,\, R_{\text{diff}})(1 - Y_B)
\end{equation}

\textbf{Fungal allocation.} Fungal yield may be constant or uptake-dependent:
%
\begin{equation}
Y_F = \text{const.} \quad \text{or} \quad Y_F = \frac{Y_{F,\max} R_F}{R_F + K_{Y,F}}
\end{equation}
%
Assimilated carbon enters the fungal biomass pools (Section~2.4.4):
%
\begin{equation}
\Gamma_F = Y_F R_F
\end{equation}
%
and respiration is:
%
\begin{equation}
\text{Resp}_F = (1 - Y_F) R_F
\end{equation}
%
The distribution of $\Gamma_F$ among the three fungal compartments is described in the following subsection.

\subsubsection{Fungal Community Dynamics}
The fungal community is modeled as an assembly of immobile and mobile pools following \citet{Falconer2005}. The non-insulated immobile pool ($F_n$) represents active hyphal tips responsible for territorial invasion and the majority of resource uptake. Over time, non-insulated tips are mechanically insulated by new growth at rate $\zeta(T) F_n$, converting to the insulated pool ($F_i$), which represents mature hyphal structures responsible for binding soil particles. The insulated pool uptakes resources at a reduced rate (controlled by $\lambda$ in the uptake equation) and is the only fungal pool that undergoes death. Resources taken up by the immobile pools are transferred immediately to the mobile pool ($F_m$), which extends throughout the hyphal network. Passive and active transport of the mobile pool is encapsulated by an effective diffusive process that delivers resources to regions of active growth.

The mobile pool can be locally immobilized to insulated or non-insulated pools at rates $\beta_i \Pi$ and $\beta_n \Pi$ respectively, while the immobile pools can be mobilized back to the mobile pool at nonlinear rates $\alpha_i \Pi^\delta$ and $\alpha_n \Pi^\delta$, where
%
\begin{equation}
\Pi = \frac{F_m}{F_i + F_n + \varepsilon_F}
\end{equation}
%
is the mobile-to-immobile biomass ratio, $\varepsilon_F = 10^{-10}\;\mu\text{g\,mm}^{-3}$ is a small regularization constant that prevents division by zero when both immobile pools are near depletion, and $\delta > 1$. The nonlinear dependence on $\Pi$ (with $\delta > 1$) ensures that mobilization accelerates relative to immobilization as $\Pi$ decreases, guaranteeing a stable lower bound on the mobile pool. Specifically, when $\Pi$ falls below the threshold $(\beta/\alpha)^{1/(\delta-1)}$, the mobilization term dominates, creating a restoring mechanism that prevents complete depletion of mobile resources at active hyphal tips.

All fungal transition rates---$\alpha_i$, $\alpha_n$, $\beta_i$, $\beta_n$, and $\zeta$---scale with temperature via a single shared activation energy $\mathcal{E}_{a,F}$ (Section~\ref{sec:temperature}), reflecting the assumption that these processes are governed by common fungal cellular machinery (cytoplasmic streaming, tip growth, cell wall deposition).

Both mobilization and immobilization carry a metabolic conversion cost: the immobile pools retain only fraction $\eta < 1$ of the exchanged biomass, with the remainder $(1-\eta)$ lost to respiration. The net transfer rates are therefore:
%
\begin{align}
F_n \to F_i: &\quad \zeta(T) F_n \\
F_m \to F_i: &\quad \eta\left(\beta_i(T) \Pi - \alpha_i(T) \Pi^\delta\right) F_i \\
F_m \to F_n: &\quad \eta\left(\beta_n(T) \Pi - \alpha_n(T) \Pi^\delta\right) F_n \\
\text{Resp}_F^{\text{conv}}: &\quad (1-\eta)\left[\left|\left(\beta_i \Pi - \alpha_i \Pi^\delta\right) F_i\right| + \left|\left(\beta_n \Pi - \alpha_n \Pi^\delta\right) F_n\right|\right]
\end{align}
%
where positive values indicate net immobilization and negative values indicate net mobilization. Assimilated fungal carbon $\Gamma_F$ enters $F_m$ directly. The respiration cost $\text{Resp}_F^{\text{conv}}$ contributes to total fungal respiration alongside $\text{Resp}_F$ and to oxygen consumption.

\subsubsection{Biomass and EPS Turnover}

Microbial biomass and EPS are not permanent sinks for carbon. Death and degradation return organic carbon from the biotic pools to the soluble pool $C$, closing the carbon cycle within the aggregate. The three recycling fluxes are each governed by distinct mechanisms.

\textbf{Bacterial death} is first-order in biomass:
%
\begin{equation}
R_{\text{rec},B} = \mu_B(T) \, B \, h_B
\end{equation}
%
where $\mu_B(T)$ scales with temperature via $\mathcal{E}_{a,B}$ (Section~\ref{sec:temperature}) and $h_B$ is the same threshold function defined in the bacterial allocation section. The product $\mu_B B h_B$ vanishes smoothly as $B \to 0$, ensuring that the death rate cannot drive bacterial biomass negative.

\textbf{Fungal death} acts only on the insulated pool $F_i$, which represents mature hyphal structures no longer actively maintained by the network:
%
\begin{equation}
R_{\text{rec},F} = \mu_F(T) \, F_i \, h_{F_i}
\end{equation}
%
where $\mu_F(T)$ scales with temperature via $\mathcal{E}_{a,F}$ (Section~\ref{sec:temperature}) and the threshold function
%
\begin{equation}
h_{F_i} = \frac{\exp(\beta_F F_i)}{\exp(\beta_F F_i) + \exp(\beta_F F_{i,\min})}
\end{equation}
%
with $\beta_F = 50/F_{i,\min}$, is analogous to $h_B$, ensuring $R_{\text{rec},F}$ vanishes smoothly as $F_i \to 0$. The non-insulated and mobile pools are sustained by exchanges within the hyphal network and do not undergo independent death.

\textbf{EPS degradation} is enzymatic recycling of extracellular polymeric substances. Microorganisms preferentially degrade EPS when soluble carbon is scarce, and suppress degradation when alternative carbon sources are abundant \citep{Frederick2011}. This is captured by a substrate-inhibited rate:
%
\begin{equation}
R_{\text{rec},E} = \mu_E^{\max}(T) \, \frac{K_E}{K_E + C_{aq}} \, E \, h_E
\end{equation}
%
where $\mu_E^{\max}(T)$ scales with temperature via $\mathcal{E}_{a,E}$ (Section~\ref{sec:temperature}), $K_E$ is the inhibition concentration, and the threshold function
%
\begin{equation}
h_E = \frac{\exp(\beta_E E)}{\exp(\beta_E E) + \exp(\beta_E E_{\min})}
\end{equation}
%
with $\beta_E = 50/E_{\min}$, ensures $R_{\text{rec},E}$ vanishes smoothly as $E \to 0$. The rate approaches zero as $C$ becomes large relative to $K_E$.

The total recycling flux entering the soluble pool is:
%
\begin{equation}
R_{\text{rec}} = R_{\text{rec},B} + R_{\text{rec},F} + R_{\text{rec},E}
\end{equation}
%
Note that the metabolic respiration cost of fungal biomass conversion ($\text{Resp}_F^{\text{conv}}$, defined in Section~2.4.4) does not contribute to $R_{\text{rec}}$: that carbon is lost as CO$_2$ rather than returned to the soluble pool.

\subsubsection{Net Sources and Sinks}

The source and sink terms $\mathbf{S}(\mathbf{u}, r, t)$ in the master equation are assembled here from the individual process descriptions in preceding subsections. Five variables ($C$, $B$, $F_n$, $F_m$, $O$) are governed by PDEs on the domain $r_0 \leq r \leq r_{\max}$; three variables ($F_i$, $E$, $M$) are immobile and governed by local ODEs at each radial position; and $P$ is spatially uniform and governed by a single ODE.

\textbf{Particulate organic matter} is depleted by surface dissolution:
%
\begin{equation}
S_P=\frac{dP}{dt} = -R_P = -4\pi r_0^2 \, J_P
\end{equation}
\textbf{Particulate organic matter} is depleted by surface dissolution:
%
\begin{equation}
S_P = -R_P
\end{equation}

\textbf{Soluble organic carbon} is the central hub through which all carbon fluxes route. POM dissolution enters as a flux at the inner boundary:
%
\begin{equation}
-D_C \left.\frac{\partial C}{\partial r}\right|_{r = r_0} = R_P
\end{equation}
%
The volumetric source/sink term is:
%
\begin{equation}
S_C = -R_B - R_F + R_{\text{rec}} - J_M \frac{\theta + \rho_b k_d}{k_d}
\end{equation}

\textbf{Bacterial biomass} grows under favorable conditions and decays through death:
%
\begin{equation}
S_B = \Gamma_B - R_{\text{rec},B}
\end{equation}
%
where $\Gamma_B$ is defined conditionally in bacterial allocation: under growth conditions $\Gamma_B = Y_B R_{\text{diff}}(1-\gamma)$; under starvation $\Gamma_B = R_{\text{diff}} < 0$.

\textbf{Insulated immobile fungi} accumulate through tip insulation and net immobilization from the mobile pool, and are lost through death:
%
\begin{equation}
S_{F_i} = \zeta(T) F_n + \eta\left(\beta_i(T) \Pi - \alpha_i(T) \Pi^\delta\right) F_i - R_{\text{rec},F}
\end{equation}

\textbf{Non-insulated immobile fungi} gain carbon from the mobile pool and lose it through insulation:
%
\begin{equation}
S_{F_n} = \eta\left(\beta_n(T) \Pi - \alpha_n(T) \Pi^\delta\right) F_n - \zeta(T) F_n
\end{equation}

\textbf{Mobile fungi} receive all assimilated fungal carbon and redistribute it to the immobile pools, bearing conversion respiration losses:
%
\begin{equation}
S_{F_m} = \Gamma_F - \eta\left(\beta_i(T) \Pi - \alpha_i(T) \Pi^\delta\right) F_i - \eta\left(\beta_n(T) \Pi - \alpha_n(T) \Pi^\delta\right) F_n - \text{Resp}_F^{\text{conv}}
\end{equation}

\textbf{Extracellular polymeric substances} are produced during bacterial growth and degraded enzymatically:
%
\begin{equation}
S_E = \Gamma_E - R_{\text{rec},E}
\end{equation}

\textbf{Mineral-associated organic carbon} evolves according to the two-stage sorption flux:
%
\begin{equation}
S_M = J_M
\end{equation}

\textbf{Oxygen} is consumed stoichiometrically by all respiratory fluxes:
%
\begin{equation}
S_O = -\alpha_O \left[\text{Resp}_B + \text{Resp}_F + \text{Resp}_F^{\text{conv}}\right]
\end{equation}
%
where $\alpha_O$ is the respiratory quotient converting carbon oxidation to oxygen consumption.

Carbon conservation requires that total carbon input equals the sum of carbon leaving as \chem{CO_2} and carbon accumulating in all pools. This can be verified by summing the source/sink terms over all carbon-bearing variables:
%
\begin{equation}
S_P + S_C + S_B + S_{F_i} + S_{F_n} + S_{F_m} + S_E + S_M = -\left[\text{Resp}_B + \text{Resp}_F + \text{Resp}_F^{\text{conv}}\right]
\end{equation}
%
The right-hand side is the only net carbon loss from the domain, confirming that all internal fluxes—uptake, allocation, recycling, and MAOC stabilization—redistribute carbon among pools without creating or destroying it. Oxygen consumption follows stoichiometrically through $S_O = -\alpha_O$ times the same respiratory sum.


\subsubsection{Transport Processes}

Five of the nine state variables exhibit spatial transport through diffusive processes: soluble organic carbon ($C$), bacterial biomass ($B$), non-insulated immobile fungi ($F_n$), mobile fungi ($F_m$), and oxygen ($O$). The remaining variables are immobile: particulate organic matter ($P$) and mineral-associated organic carbon ($M$) are solid phases, extracellular polymeric substances ($E$) adhere to particle surfaces, and insulated immobile fungi ($F_i$) form stationary hyphal structures.

The effective diffusion coefficients account for tortuosity, phase partitioning, and biological constraints on mobility. The Millington-Quirk tortuosity model $\tau = \theta^{2}/\theta_s^{2/3}$ represents increased path length and reduced cross-sectional area for transport through water-filled pore space. The tortuosity factors $\tau(\theta)$ and the Millington--Quirk exponents are temperature-independent, as they reflect pore geometry rather than molecular kinetics.

The diffusion of \textbf{soluble organic carbon} is retarded by partitioning between dissolved and equilibrium-sorbed phases:
%
\begin{equation}
D_C = D_{C0}(T) \, \tau \, \frac{\theta}{\theta + \rho_b k_d}
\end{equation}
%
where $D_{C0}(T)$ is the temperature-dependent molecular diffusion coefficient of dissolved organic carbon in water (Section~\ref{sec:temperature}), and the factor $\theta/(\theta + \rho_b k_d)$ accounts for the fraction of carbon instantaneously sorbed to mineral surfaces with partition coefficient $k_d$.

\textbf{Bacterial motility} is represented as diffusion proportional to carbon transport, reflecting chemotactic movement in response to substrate gradients:
%
\begin{equation}
D_B = D_{B,\text{rel}} \, D_C
\end{equation}
%
where $D_{B,\text{rel}} \approx 0.5$ based on experimental observations of bacterial diffusivity relative to dissolved organic substrates. Bacterial motility inherits temperature dependence from $D_C$.

\textbf{Fungal biomass}: Non-insulated fungal transport represents hyphal tip extension through water-filled pore space:
%
\begin{equation}
D_{F_n} = D_{F_n,0}(T) \, \tau
\end{equation}
%
where $D_{F_n,0}(T)$ scales with temperature via $\mathcal{E}_{a,F}$ (Section~\ref{sec:temperature}), reflecting the biological nature of hyphal extension. Mobile fungal transport represents internal resource translocation within the hyphal network:
%
\begin{equation}
D_{F_m} = D_{F_m,0}(T)
\end{equation}
%
where $D_{F_m,0}(T)$ scales with temperature via $\mathcal{E}_{a,F}$. Mobile fungal transport carries no tortuosity factor: internal translocation occurs within the hyphal network rather than through the external pore space, allowing mobile fungi to transport resources through regions of low water content.

\textbf{Oxygen} transport occurs through both aqueous and gaseous phases in parallel:
%
\begin{equation}
D_O = D_{DO} + D_{GO}
\end{equation}
%
where the aqueous and gaseous contributions are:
%
\begin{align}
D_{DO} &= D_{DO0}(T) \, \frac{\theta}{\theta + K_H(T) \theta_a} \, \frac{\theta^2}{\theta_s^{2/3}} \\
D_{GO} &= D_{GO0}(T) \, \frac{K_H(T) \theta_a}{\theta + K_H(T) \theta_a} \, \frac{\theta_a^{10/3}}{\theta_s^2}
\end{align}
%
The prefactors partition oxygen between phases according to the temperature-dependent Henry's law constant $K_H(T)$ (Section~\ref{sec:temperature}), while distinct tortuosity functions reflect differing sensitivities to pore connectivity: the aqueous phase follows Millington-Quirk ($\theta^2/\theta_s^{2/3}$), while the gaseous phase exhibits stronger dependence on air-filled porosity with exponent $10/3$ reflecting the greater sensitivity of gas diffusion to pore connectivity. Both $D_{DO0}(T)$ and $D_{GO0}(T)$ are temperature-dependent: the former follows an empirical correlation for O$_2$ diffusion in water, and the latter follows Chapman--Enskog kinetic theory for binary gas mixtures (Section~\ref{sec:temperature}).

\subsubsection{Water Retention and EPS Modification}

Volumetric water content is derived from matric potential via the van Genuchten retention curve \citep{vanGenuchten1980}:
%
\begin{equation}
\theta = \theta_r + (\theta_s - \theta_r)\left(1 + (\alpha\psi)^n\right)^{(1/n)-1}
\end{equation}
%
where $\theta_r$ and $\theta_s$ are residual and saturated water contents, $1/\alpha$ is the matric potential at which maximum drainage occurs, and $n$ is a pore-size distribution shape parameter.

Accumulation of hydrophilic EPS and fungal hyphae alters water retention locally by tightening pores and increasing surface wettability \citep{Ghezzehei2015}. This is captured by modifying the van Genuchten $\alpha$ parameter:
%
\begin{equation}
\alpha = \alpha_0 \, \exp\!\left(\omega_E E + \omega_F F_i\right)
\end{equation}
%
where $\alpha_0$ is the unaltered parameter for bare mineral surfaces, and $\omega_E$, $\omega_F$ are sensitivity coefficients for EPS and insulated fungal hyphae respectively. Both $\omega_E$ and $\omega_F$ are negative: increasing EPS or $F_i$ reduces $\alpha$, shifting the retention curve toward higher water contents at a given $\psi$. The result is spatially heterogeneous water retention that evolves with the domain life cycle — regions colonized by fungi or enriched in EPS retain more water than the outer matrix, creating feedback between biological activity and moisture distribution.

\subsubsection{Temperature Dependence}
\label{sec:temperature}

All rate processes and transport properties depend on temperature. We distinguish three mechanisms: Arrhenius kinetics for biologically mediated reactions, molecular transport theory for diffusion coefficients, and thermodynamic equilibrium for gas--liquid partitioning. Each is described from first principles rather than through empirical $Q_{10}$ factors.

\paragraph{Biological rate constants.}
All biological rate parameters $k$ specified at reference temperature $T_{\text{ref}}$ are scaled via the Arrhenius equation:
%
\begin{equation}
k(T) = k_{\text{ref}} \exp\!\left[\frac{\mathcal{E}_a}{R}\left(\frac{1}{T_{\text{ref}}} - \frac{1}{T}\right)\right]
\label{eq:arrhenius}
\end{equation}
%
where $\mathcal{E}_a$ is the activation energy [J\,mol$^{-1}$] and $R = 8.314$\,J\,mol$^{-1}$\,K$^{-1}$ is the universal gas constant. Six distinct activation energies capture the different thermal sensitivities of biological and abiotic processes (Table~\ref{tab:activation_energies}).

\begin{table}[h]
\caption{Activation energies for temperature-dependent rate processes. The column ``Rate parameters'' lists all model quantities governed by each activation energy. Default values represent central estimates from the literature ranges cited; all are subject to calibration.}
\label{tab:activation_energies}
\centering
\begin{tabular}{llcll}
\hline
Process & Symbol & $\mathcal{E}_a$ [kJ\,mol$^{-1}$] & Rate parameters & Rationale \\
\hline
Bacterial metabolism & $\mathcal{E}_{a,B}$ & 60 & $r_{B,\max}$, $\mu_B$ & Enzyme-catalyzed carbon oxidation \\
Fungal metabolism & $\mathcal{E}_{a,F}$ & 55 & $r_{F,\max}$, $\mu_F$, $\alpha_i$, $\alpha_n$, $\beta_i$, $\beta_n$, $\zeta$, $D_{F_n,0}$, $D_{F_m,0}$ & Shared fungal cellular machinery \\
EPS degradation & $\mathcal{E}_{a,E}$ & 50 & $\mu_E^{\max}$ & Extracellular enzymatic hydrolysis \\
MAOC sorption & $\mathcal{E}_{a,s}$ & 25 & $\kappa_s$ & Abiotic surface adsorption \\
MAOC desorption & $\mathcal{E}_{a,d}$ & 40 & $\kappa_d$ & Bond-breaking from mineral surfaces \\
POM dissolution & $\mathcal{E}_{a,P}$ & 60 & $R_P^{\max}$ & Enzymatic surface attack \\
\hline
\end{tabular}
\end{table}

The activation energies for bacterial and fungal metabolism are consistent with reported ranges of 40--80\,kJ\,mol$^{-1}$ for microbial respiration in soils \citep{Sierra2015,Davidson2006}. All fungal rate parameters---growth, mortality, transition rates ($\alpha_i$, $\alpha_n$, $\beta_i$, $\beta_n$, $\zeta$), and translocation diffusivities ($D_{F_n,0}$, $D_{F_m,0}$)---share a single activation energy $\mathcal{E}_{a,F}$, reflecting the assumption that these processes are governed by common fungal cellular machinery (cytoplasmic streaming, tip growth, cell wall deposition). This avoids introducing unconstrained parameters for each transition rate.

MAOC sorption and desorption activation energies reflect the distinction between physical adsorption ($\mathcal{E}_{a,s}$, governed by diffusion to surface sites) and chemical desorption ($\mathcal{E}_{a,d}$, requiring disruption of organo-mineral bonds through ligand exchange or protonation). The inequality $\mathcal{E}_{a,d} > \mathcal{E}_{a,s}$ has an important consequence: warming preferentially accelerates desorption relative to sorption, narrowing the hysteresis between the sorption and desorption pathways described in Section~2.4.2. This creates a testable prediction that MAOC persistence decreases with temperature not only through enhanced microbial mineralization but also through reduced kinetic trapping at mineral surfaces.

\paragraph{Diffusion in water.}
Molecular diffusion coefficients in water follow from the Stokes--Einstein relation:
%
\begin{equation}
D_w(T) = \frac{k_B\, T}{6\pi\, \eta(T)\, a}
\label{eq:stokes_einstein}
\end{equation}
%
where $k_B$ is Boltzmann's constant, $\eta(T)$ is the dynamic viscosity of water, and $a$ is the solute hydrodynamic radius. Since $a$ is temperature-independent, the ratio relative to a reference measurement simplifies to:
%
\begin{equation}
D_w(T) = D_{w,\text{ref}} \cdot \frac{T}{T_{\text{ref}}} \cdot \frac{\eta(T_{\text{ref}})}{\eta(T)}
\label{eq:D_water_ratio}
\end{equation}
%
The temperature dependence of water viscosity is described by the Vogel--Fulcher--Tammann (VFT) equation:
%
\begin{equation}
\ln\!\left[\eta(T)\,/\,\text{mPa\,s}\right] = A + \frac{B}{T - T_0}
\label{eq:VFT}
\end{equation}
%
with $A = -3.7188$, $B = 578.919$\,K, and $T_0 = 137.546$\,K, valid over the range 273--373\,K \citep{Vogel1921,Fulcher1925}. This formulation applies to dissolved organic carbon ($D_{C0}$), which is well described by Stokes--Einstein for sugar-sized molecules in dilute aqueous solution.

For dissolved oxygen, the Stokes--Einstein relation shows systematic deviations over wide temperature ranges. We adopt the empirical correlation of \citet{Han1996}, which was fitted to Taylor dispersion measurements from $-0.5$ to $95\,^{\circ}$C:
%
\begin{equation}
\log_{10}\!\left[\frac{D_{O_2,w}}{\text{cm}^2\,\text{s}^{-1}}\right] = -4.410 + \frac{773.8}{T} - \left(\frac{506.4}{T}\right)^{\!2}
\label{eq:han_bartels}
\end{equation}
%
where $T$ is in Kelvin. At $T = 293.15$\,K, this yields $D_{O_2,w} = 2.01 \times 10^{-5}$\,cm$^2$\,s$^{-1}$ ($= 173.7$\,mm$^2$\,day$^{-1}$).

\paragraph{Diffusion in air.}
Gas-phase diffusion coefficients follow from Chapman--Enskog kinetic theory for binary gas mixtures \citep{Bird2002}:
%
\begin{equation}
D_a(T) = D_{a,\text{ref}} \left(\frac{T}{T_{\text{ref}}}\right)^{1.75}
\label{eq:chapman_enskog}
\end{equation}
%
The exponent 1.75 arises from the Lennard-Jones intermolecular potential and is a well-established approximation for nonpolar gas pairs at moderate pressures. This applies to oxygen in air ($D_{GO0}$). Gas-phase DOC transport is negligible and not modeled.

\paragraph{Henry's law constant.}
The dimensionless Henry's law constant governing gas--liquid partitioning of oxygen depends on temperature through the van't Hoff relation:
%
\begin{equation}
K_H(T) = K_{H,\text{ref}} \exp\!\left[-\frac{\Delta H_{\text{sol}}}{R}\left(\frac{1}{T} - \frac{1}{T_{\text{ref}}}\right)\right]
\label{eq:henry_vanthoff}
\end{equation}
%
where $\Delta H_{\text{sol}} \approx -12$\,kJ\,mol$^{-1}$ is the enthalpy of dissolution of O$_2$. The negative sign reflects the exothermic nature of O$_2$ dissolution: warming decreases solubility (increases $K_H$), reducing the aqueous oxygen fraction and enhancing gas-phase transport. This creates a compensating effect on oxygen supply to the aggregate interior: warming increases both diffusion coefficients and gas partitioning, potentially reducing the extent of anoxic zones.

\paragraph{Effective diffusion coefficients.}
The effective diffusion coefficients defined in Section~2.4.7 are now understood as functions of temperature through their dependence on the pure-phase coefficients:
%
\begin{align}
D_C(T,\theta) &= D_{C0}(T) \; \tau(\theta) \; \frac{\theta}{\theta + \rho_b k_d} \\[4pt]
D_O(T,\theta) &= D_{O_2,w}(T) \; \frac{\theta}{\theta + K_H(T)\,\theta_a} \; \frac{\theta^2}{\theta_s^{2/3}} \;+\; D_{O_2,a}(T) \; \frac{K_H(T)\,\theta_a}{\theta + K_H(T)\,\theta_a} \; \frac{\theta_a^{10/3}}{\theta_s^2}
\end{align}
%
The tortuosity factors $\tau(\theta)$ and the Millington--Quirk exponents remain temperature-independent, as they reflect pore geometry rather than molecular kinetics.

Bacterial motility ($D_B$), fungal hyphal extension ($D_{F_n}$), and internal fungal translocation ($D_{F_m}$) inherit temperature dependence through their coupling to $D_C$ or through biological rate scaling:
%
\begin{align}
D_B(T,\theta) &= D_{B,\text{rel}} \; D_C(T,\theta) \\
D_{F_n}(T,\theta) &= D_{F_n,0}(T) \; \tau(\theta) \\
D_{F_m}(T) &= D_{F_m,0}(T)
\end{align}
%
where $D_{F_n,0}(T)$ and $D_{F_m,0}(T)$ scale with temperature via Eq.~\eqref{eq:arrhenius} using $\mathcal{E}_{a,F}$, reflecting the biological nature of hyphal extension and translocation.

\subsection{Aggregate Stability Mechanics}

Aggregate stability arises from binding forces exerted by EPS and fungal hyphae that must exceed hydrodynamic stresses during disaggregation events such as wet sieving. We derive the critical binding strength from first principles using Stokes flow theory for low Reynolds number conditions.



For a spherical aggregate of radius $r$ moving through viscous fluid with velocity $v$, the drag force follows Stokes' law:
%
\begin{equation}
F_d = 6\pi \mu\, v\, r
\end{equation}
%
where $\mu$ is the dynamic viscosity of water. The corresponding drag stress on the aggregate surface is:
%
\begin{equation}
\sigma_d = \frac{F_d}{4\pi r^2} = \frac{3\mu v}{2r}
\end{equation}
%
The cohesive stress from binding agents is a linear combination of fungal and EPS contributions:
%
\begin{equation}
\sigma_{\text{cohesive}} = k_F F_i + k_E E
\end{equation}
%
where $k_F$ and $k_E$ are specific cohesive strengths. The specific cohesive strength of fungal hyphae is derived from the mechanical properties of individual strands:
%
\begin{equation}
k_F = \frac{\sigma_h}{\rho_h}
\end{equation}
%
where $\sigma_h$ is hyphal tensile strength and $\rho_h$ is hyphal carbon density. EPS binding is assumed half as effective as fungal enmeshment: $k_E = k_F/2$.

Surface roughness from soil texture is accounted for through an effective radius:
%
\begin{equation}
r_{\text{eff}} = r + \chi\, a_p
\end{equation}
%
where $a_p$ is the median primary particle diameter and $\chi$ is a roughness amplification factor. The stability criterion requires cohesive stress to exceed hydrodynamic stress:
%
\begin{equation}
k_F\left(F_i + \tfrac{1}{2} E\right) \geq \frac{3\mu v}{2\,r_{\text{eff}}}
\end{equation}
%
Rearranging yields the critical condition:
%
\begin{equation}
\left(F_i + \tfrac{1}{2} E\right) r_{\text{eff}} \geq G_c = \frac{3\mu v}{2\,k_F}
\end{equation}
%

The characteristic velocity during wet sieving is estimated from standard sieve shaker specifications. For stroke length $L$ and frequency $f$, the maximum velocity is $v_{\max} = \pi f L$. The stable aggregate radius at time $t$ is therefore:
%
\begin{equation}
r_{\text{agg}}(t) = \max\left\{r : \left(F_i(r,t) + \tfrac{1}{2}E(r,t)\right)(r + \chi\, a_p) \geq G_c\right\}
\end{equation}
%
The radius $r_{\text{agg}}(t)$ is diagnostic: it does not introduce a discontinuity in PDE coefficients or act as a transport boundary, but serves solely as the integration limit for the dual integration domains defined in Section~2.3.2. Carbon within $r_{\text{agg}}(t)$ would survive wet sieving and contribute to aggregate-associated pools, while carbon beyond it remains in the intact soil matrix.

\subsection{Boundary Conditions}

The biogeochemical domain is bounded by two interfaces: the POM surface at $r = r_0$ and the outer boundary at $r = r_{\max}$ connecting to the bulk soil matrix.

\textbf{Soluble organic carbon} is the central hub through which all carbon fluxes route. POM dissolution enters as a flux at the inner boundary ($r = r_0$):
%
\begin{equation}
-D_C \left.\frac{\partial C}{\partial r}\right|_{r = r_0} = J_P
\end{equation}
%
All other mobile variables have zero-flux (Neumann) conditions at the POM surface: $\partial B/\partial r = \partial F_n/\partial r = \partial F_m/\partial r = \partial O/\partial r = 0$ at $r = r_0$.

\textbf{Outer boundary ($r = r_{\max}$).} The domain connects to the bulk soil matrix at $r = r_{\max}$. Oxygen equilibrates with the ambient concentration:
%
\begin{equation}
O(r_{\max}, t) = O_{\text{amb}}(t)
\end{equation}
%
Biological variables ($B$, $F_n$, $F_m$) and dissolved carbon ($C$) have zero-flux conditions, representing negligible exchange with the surrounding matrix:
%
\begin{equation}
\frac{\partial B}{\partial r}\bigg|_{r=r_{\max}} = \frac{\partial F_n}{\partial r}\bigg|_{r=r_{\max}} = \frac{\partial F_m}{\partial r}\bigg|_{r=r_{\max}} = \frac{\partial C}{\partial r}\bigg|_{r=r_{\max}} = 0
\end{equation}
%
The radius $r_{\max}$ is chosen sufficiently large (typically 5-10 times the initial POM radius) such that biogeochemical activity decays to near-background levels and boundary effects do not influence the domain interior.

\section{Model Evaluation Against Experimental Data}

The framework is evaluated against time-series data from controlled laboratory incubations and long-term field observations. Model parameters governing biogeochemical rates were adjusted to match observed aggregate mass and respiration dynamics, providing fitted representations of soil-specific reactivity and organic matter input rates. These evaluations establish the model's capacity to reproduce macroscopic observables across timescales ranging from weeks to decades, building confidence for the mechanistic predictions presented in Section~4.

\subsection{Laboratory Incubations}

We evaluated the model against two laboratory incubation studies in which soils were amended with crushed organic matter and tracked over contrasting timescales. 

The short-term study \citep{DeGryze2006} used a silt loam soil (18\% clay, 56\% silt, 26\% sand) amended with crushed wheat straw at a rate of 5 g kg$^{-1}$ soil. Samples were incubated at constant water content ($\theta = 0.25$ m$^3$ m$^{-3}$) and temperature (20°C) for 21 days. Aggregate stability was quantified by mean weight diameter (MWD) following wet sieving, and cumulative CO$_2$ production was monitored throughout.

The long-term study \citep{Pronk2012} used a sandy loam soil (12\% clay, 20\% silt, 68\% sand) amended with crushed cattle manure at a rate of 10 g C kg$^{-1}$ soil. Samples were incubated at constant water content ($\theta = 0.20$ m$^3$ m$^{-3}$) and temperature (20°C) for 18 months. The mass of water-stable aggregates $>250$ $\mu$m was determined by wet sieving at multiple time points, and cumulative CO$_2$ production was monitored.

For each study, the POM size distribution was estimated from the reported particle size of the crushed organic amendments. Biogeochemical rate constants were adjusted by a uniform scaling factor (reactivity rate) to match observed aggregate mass and cumulative respiration. This reactivity rate represents site-specific factors not explicitly modeled, including organic matter chemistry, microbial community composition, soil mineralogy, and nutrient availability. The fitted reactivity rates were 1.2 for the wheat straw experiment and 0.8 for the manure experiment, reflecting differences in substrate quality and soil properties between the two studies. 

Both incubations were conducted at constant temperature (20\,$^{\circ}$C $= 293.15$\,K), which coincides with the reference temperature $T_{\text{ref}}$. Consequently, all Arrhenius factors reduce to unity, and the temperature framework introduced in Section~\ref{sec:temperature} does not affect the calibrated fits. Validation of the temperature response requires experimental data spanning multiple temperatures, which is an important next step.

\begin{figure}
% PLACEHOLDER: Two-panel figure
% Panel A: Short-term (wheat straw, 21 days)
%   - X-axis: Time (days, 0-21)
%   - Left Y-axis: Mean weight diameter (mm)
%   - Right Y-axis: Cumulative CO2-C (mg/g soil)
%   - Plot experimental data points (symbols) with error bars
%   - Overlay model predictions (lines) for both MWD and respiration
% Panel B: Long-term (manure, 18 months)
%   - X-axis: Time (months, 0-18)
%   - Left Y-axis: Aggregated volume (% of total)
%   - Right Y-axis: Cumulative CO2-C (mg/g soil)
%   - Plot experimental data points with error bars
%   - Overlay model predictions for both variables
\caption{Model evaluation against laboratory incubation experiments. (a) Short-term incubation with crushed wheat straw: mean weight diameter (MWD) and cumulative respiration over 21 days. (b) Long-term incubation with crushed manure: aggregated soil volume and cumulative respiration over 18 months. Symbols show experimental data; lines show model predictions. Both experiments exhibit rapid initial aggregation followed by respiratory decline as resources are depleted.}
\label{fig:incubations}
\end{figure}

Both experiments and model predictions show rapid aggregation within weeks of organic matter addition (Figure~\ref{fig:incubations}), consistent with observations from similar short-term studies \citep{Andruschkewitsch2014}. In the wheat straw experiment, aggregation increased nearly linearly over the 21-day period. Model extrapolation beyond the experimental duration predicts that aggregation would have begun to plateau around 5 weeks as the initial pulse of easily decomposable carbon is exhausted. The manure experiment captured the full aggregation cycle: rapid initial increase, a peak at approximately 6 months, followed by gradual decline as POM is depleted and binding agents degrade. Cumulative respiration in both studies tracked the progressive exhaustion of bioavailable carbon, with the model capturing both the initial rapid mineralization phase and the subsequent exponential decline.

The close correspondence between model predictions and observations across two orders of magnitude in timescale demonstrates the framework's capacity to represent aggregate formation dynamics driven by organic matter inputs under controlled conditions.

\subsection{Long-Term Field Restoration Studies}

To evaluate the model's capacity to represent multi-decadal aggregation dynamics under natural conditions, we used time-series data from three long-term field monitoring sites where aggregation recovery was tracked following conversion of tilled agricultural land to perennial grassland. The three sites span contrasting climatic and edaphic conditions: Fermi National Accelerator Laboratory in Illinois, USA \citep{Jastrow1987}, Jealott's Hill Experimental Station in Berkshire, UK \citep{Low1955}, and Waite Research Institute in South Australia \citep{Greacen1958}. In all three studies, the proportion of water-stable aggregates was measured periodically over 10-15 years following cessation of tillage.

For each site, two parameters were adjusted to match observed aggregation trajectories: (1) the annual volumetric POM input rate, representing root turnover and litter deposition, and (2) a uniform scaling factor for biogeochemical rate constants (reactivity rate), representing site-specific controls on decomposition and microbial activity. The fitted POM input rates were 0.48\%, 3\%, and 6.8\% of total soil volume per year for the US, UK, and Australia sites, respectively. The corresponding reactivity rates were 0.25, 0.75, and 1.0. A strong positive correlation between these two fitted parameters is evident: environments that support higher root biomass production also exhibit faster POM turnover, consistent with the coupled nature of plant productivity and microbial activity.

\begin{figure}
% PLACEHOLDER: Two-panel figure
% Panel A: Aggregation recovery trajectories
%   - X-axis: Years since restoration (0-15 years)
%   - Y-axis: Water-stable aggregates (% of total soil volume)
%   - Three curves (US, UK, Australia) — data points (symbols) + model fits (lines)
%   - All three show rapid initial increase, approaching asymptote within ~10 years
%   - Australia site reaches highest steady-state level (~70%), US lowest (~35%)
% Panel B: Resilience to disturbance
%   - X-axis: Years (0-40, with disturbance imposed at year 20)
%   - Y-axis: Water-stable aggregates (% of total)
%   - Three curves showing response to 80% reduction in POM input at year 20
%   - Australia site (fastest aggregation) shows steepest decline post-disturbance
%   - US site (slowest aggregation) shows most gradual decline
\caption{Long-term field restoration studies. (a) Aggregation recovery following conversion from tilled agriculture to perennial grassland at three sites spanning contrasting climates and soils. Symbols show measured water-stable aggregate fractions; lines show fitted model trajectories. All three sites exhibit rapid initial recovery, reaching steady state within approximately one decade. (b) Model-predicted resilience to disturbance. An 80\% reduction in annual POM input imposed at year 20 causes rapid decline in aggregation, with the fastest-aggregating site (Australia) showing the steepest collapse. This demonstrates that rapid aggregation correlates with high vulnerability to perturbation.}
\label{fig:field_restoration}
\end{figure}

All three datasets and model simulations exhibit a characteristic temporal pattern: rapid initial aggregation following tillage cessation, reaching a long-term steady state within approximately one decade (Figure~\ref{fig:field_restoration}a). The time to steady state is inversely correlated with site reactivity — the Australian site, with the highest reactivity, reaches equilibrium most rapidly. This pattern is consistent with the soil carbon saturation concept \citep{Six2002,Stewart2007,West2007}, which posits that organic carbon accumulation approaches an asymptotic maximum following disturbance cessation. At steady state, the degree of aggregation reflects a dynamic equilibrium between continuous formation of new aggregates around fresh POM inputs and the disappearance of older aggregates as their POM cores are exhausted and binding agents degrade.

The fitted models were used to explore system resilience by simulating an 80\% reduction in annual POM input beginning at year 20 (Figure~\ref{fig:field_restoration}b). All three sites exhibit rapid decline in aggregate stability following the disturbance, but the magnitude and rate of decline differ markedly. The Australian site, which exhibited the fastest aggregation and youngest mean aggregate age at steady state, shows the steepest collapse. In contrast, the US site, with slower aggregation dynamics and older aggregate populations, exhibits more gradual decline. This counterintuitive result — that faster-aggregating systems are more vulnerable to disturbance — reflects the age structure of aggregate populations at steady state: systems with rapid turnover maintain predominantly young aggregates that depend on continuous fresh inputs, while systems with slower turnover accumulate a greater proportion of older, more stable aggregates that buffer against perturbation.

The close agreement between model predictions and field observations across sites with two-fold differences in steady-state aggregation levels and climatic conditions establishes the framework's capacity to represent long-term aggregation dynamics under natural vegetation. These validations provide confidence for the mechanistic predictions presented in Section~4.

\section{Testable Predictions from Internal Dynamics}

The evaluations in Section~3 establish the model's capacity to reproduce observed macroscopic behavior — aggregate mass and respiration trajectories — across timescales from weeks to decades. Having demonstrated this credibility, we now present predictions about internal dynamics and population-scale behavior that current experimental methods do not routinely measure but are amenable to targeted investigation. These predictions fall into four categories: carbon fate and pool partitioning, the temporal relationship between stability and biogeochemical activity, size-dependent aggregate life cycles, and the age structure of aggregate populations. Each subsection concludes with an explicit statement of testable hypotheses that could be verified through coordinated measurements combining wet sieving, carbon fractionation, isotopic labeling, and microbial community profiling.

\subsection{Carbon Fate and Pool Partitioning}

A central question in soil carbon cycling is how organic matter inputs are partitioned among transient bioavailable pools, active microbial biomass, and stabilized mineral-associated forms. Standard fractionation protocols can separate these pools, but temporal dynamics are rarely resolved at sub-annual timescales, particularly for individual aggregate size classes. The model explicitly tracks carbon partitioning throughout the aggregate life cycle, predicting how POM-derived carbon flows through the biogeochemical cascade and ultimately partitions between pools that turn over on timescales of days to weeks (DOC, microbial biomass) versus those that persist for years to decades (MAOC).

\begin{figure}
% PLACEHOLDER: Three-panel figure
% Panel A: Time evolution of all carbon pools
%   - X-axis: Time (months, 0-60)
%   - Y-axis: Carbon content (mg C per aggregate)
%   - Multiple curves showing P, C, M, B, F_total (F_i+F_n+F_m), E
%   - Representative domain: 0.5 mm initial POM, ψ=-33 kPa, [O2]=21%
%   - Show: P declines exponentially, C peaks early (~6 months) then declines
%   - M increases throughout, continuing to accumulate after P is exhausted
%   - Biomass (B+F) peaks around 12-18 months, then declines
%   - EPS follows similar trajectory to biomass but with slight lag
% Panel B: MAOC accumulation vs. aggregate diameter
%   - X-axis: Time (months, 0-60)
%   - Left Y-axis: MAOC content (mg C)
%   - Right Y-axis: Aggregate diameter (mm)
%   - Two curves on same plot
%   - Show: Aggregate diameter peaks around 18-24 months, then declines
%   - MAOC continues increasing well beyond aggregate diameter peak
%   - Critical observation: MAOC accumulation decoupled from aggregate mass
% Panel C: Predicted fractionation at key time points
%   - X-axis: Time points (6, 12, 24, 36, 48 months)
%   - Y-axis: Carbon distribution (% of initial POM-C)
%   - Stacked bar chart showing POM, DOC, MAOC, Biomass, EPS, Respired
%   - Show progressive shift from POM-dominated (6 mo) to MAOC-dominated (48 mo)
%   - By 48 months: ~60% respired, ~30% MAOC, ~5% residual POM, ~5% other
\caption{Carbon fate and pool partitioning throughout the aggregate life cycle. (a) Temporal evolution of all carbon pools for a representative domain (0.5 mm initial POM). POM declines exponentially; dissolved organic carbon peaks early then declines; MAOC accumulates continuously. (b) MAOC accumulation continues after aggregate diameter peaks and begins to decline, demonstrating that aggregate stability and long-term carbon storage are decoupled processes operating on different timescales. (c) Predicted carbon distribution at key time points, showing what a researcher would measure through sequential fractionation. By 48 months, MAOC accounts for approximately 75\% of remaining carbon, even as aggregate mass has declined from its peak.}
\label{fig:carbon_fate}
\end{figure}

The model predicts a characteristic temporal sequence of carbon transformations (Figure~\ref{fig:carbon_fate}a). POM declines exponentially as enzymatic and passive dissolution processes release soluble carbon into the surrounding matrix. Dissolved organic carbon peaks within the first 6-12 months as POM decomposition rates are highest, then declines as the substrate is exhausted. Microbial biomass (bacteria and fungi combined) follows the dissolved carbon trajectory with a lag of several months, reflecting the time required for population growth. EPS production tracks bacterial biomass, peaking around 12-18 months then declining as bacteria enter starvation. Fungal biomass exhibits a broader, more sustained peak due to slower growth rates and lower turnover compared to bacteria.

Critically, MAOC accumulation exhibits fundamentally different temporal dynamics than the labile pools (Figure~\ref{fig:carbon_fate}b). While aggregate diameter — determined by binding agent concentrations — peaks around 18-24 months then declines, MAOC continues to accumulate well beyond this point. By 48 months, aggregate diameter has returned to near its initial value as EPS and fungal hyphae have degraded, yet MAOC content is still increasing. This decoupling occurs because MAOC formation is driven by the two-stage sorption mechanism: transient pulses of elevated $C_{\text{eq}}$ during wetting-drying cycles continue to drive slow stabilization onto mineral surfaces even after the main pulse of microbial activity has subsided. The implication is that aggregate stability — the property measured by wet sieving — is not a reliable indicator of long-term carbon storage capacity.

Figure~\ref{fig:carbon_fate}c shows the predicted carbon distribution at five time points, simulating what a researcher would measure through sequential density or chemical fractionation. At 6 months, POM still accounts for approximately 40\% of remaining carbon, with DOC and microbial biomass accounting for another 20\%. By 24 months — near the peak of aggregate stability — POM has declined to 10\% and MAOC has risen to 50\% of remaining carbon. By 48 months, MAOC dominates at 75\% of remaining carbon, with only trace amounts of POM and bioavailable pools. Cumulative respiration accounts for approximately 60\% of initial POM carbon by this point, consistent with observed mineralization rates for plant-derived organic matter in temperate soils.

\textbf{Testable predictions:}
\begin{itemize}
\item The ratio of MAOC to POM should increase monotonically throughout the aggregate life cycle, reaching values of 3:1 or higher by 36-48 months post-formation, even in aggregates that have declined in stability.
\item Peak MAOC accumulation should occur 12-24 months after peak aggregate stability, creating a temporal offset between measured aggregate mass and carbon storage capacity.
\item Sequential fractionation of size-separated aggregates at multiple time points following organic matter amendment should reveal the predicted temporal sequence: early POM dominance → intermediate mixed pools → late MAOC dominance.
\item Isotopic labeling experiments ($^{13}$C or $^{14}$C) should show continued incorporation of labeled carbon into MAOC fractions long after labeled carbon has disappeared from labile pools and aggregate stability has declined.
\end{itemize}

\subsection{Stability Dynamics and Binding Agent Evolution}

A fundamental question in aggregate formation is the temporal relationship between carbon mineralization and binding agent production. Conceptual models often assume that peak microbial activity coincides with peak aggregate stability, yet the model predicts a more complex relationship: aggregate stability lags peak respiration by months to years, depending on the relative contributions of bacterial EPS versus fungal hyphae. This temporal offset has not been systematically documented because most aggregation studies either measure stability at a single endpoint or track respiration and aggregation separately without coordinated high-temporal-resolution sampling.

\begin{figure}
% PLACEHOLDER: Two-panel figure
% Panel A: Respiration vs. aggregate diameter trajectories
%   - X-axis: Time (months, 0-48)
%   - Left Y-axis: CO2 flux (mg C / day)
%   - Right Y-axis: Aggregate diameter (mm)
%   - Two curves on same plot
%   - Representative domain: 0.5 mm POM, ψ=-33 kPa
%   - Show: CO2 flux peaks around 3-6 months, declines rapidly
%   - Aggregate diameter peaks around 18-24 months (12-18 month lag)
%   - Annotate the temporal offset with arrow or shaded region
% Panel B: Binding agent contributions over time
%   - X-axis: Time (months, 0-48)
%   - Y-axis: Contribution to stability (equivalent mm of aggregate radius)
%   - Two stacked areas or separate curves: EPS contribution, F_i contribution
%   - Show: EPS dominates early (peak 6-12 months), declines rapidly
%   - F_i emerges slowly, peaks later (18-30 months), persists longer
%   - Total stability is sum of both contributions
%   - Crossover point around 12-18 months where F_i overtakes EPS
\caption{Temporal relationship between biogeochemical activity and aggregate stability. (a) CO$_2$ flux peaks within 3-6 months as POM decomposition rates are highest, but aggregate diameter does not peak until 18-24 months later. This lag reflects the time required for slower-growing fungi to accumulate sufficient hyphal biomass to dominate binding. (b) Contributions of EPS and fungal hyphae ($F_i$) to aggregate stability over time. Bacterial EPS dominates initially, producing rapid early aggregation, but degrades quickly as bacteria enter starvation. Fungal hyphae accumulate more slowly but persist longer due to lower turnover rates and internal resource translocation, sustaining aggregate stability well beyond the peak of microbial activity.}
\label{fig:stability_lag}
\end{figure}

The model predicts a pronounced temporal offset between peak respiration and peak aggregate stability (Figure~\ref{fig:stability_lag}a). For a representative 0.5 mm POM under moderately moist conditions, CO$_2$ flux peaks within 3-6 months as POM decomposition rates are highest and microbial populations are growing exponentially. However, aggregate diameter does not reach its maximum until 18-24 months — a lag of 12-18 months. This offset arises from the contrasting life histories of the two dominant binding agents: bacterial EPS and fungal hyphae.

Bacterial EPS production tracks bacterial biomass closely, peaking within the first year as easily decomposable carbon is abundant (Figure~\ref{fig:stability_lag}b). This produces rapid early aggregation, consistent with the fast response observed in short-term incubation experiments. However, bacterial populations are also subject to rapid turnover and starvation as substrate availability declines. When uptake falls below basal maintenance requirements, bacteria catabolize their own biomass and cease EPS production. The result is a sharp peak in EPS-driven stability around 6-12 months, followed by rapid decline.

Fungal hyphae exhibit fundamentally different dynamics. Fungi grow more slowly than bacteria due to lower maximum specific uptake rates and the metabolic costs of hyphal extension and internal resource translocation. However, once established, hyphal networks are more resilient to substrate limitation. The three-pool fungal model allows mature insulated hyphae ($F_i$) — the primary contributors to aggregate stability — to persist by mobilizing resources from the internal mobile pool ($F_m$) even when local substrate concentrations are low. This creates a delayed but sustained contribution to stability. The crossover point where fungal binding overtakes bacterial binding occurs around 12-18 months, and fungal dominance can persist for several additional years.

The implications are significant for interpreting aggregation experiments. Short-term studies (days to weeks) capture primarily EPS-driven dynamics and may overestimate the role of bacteria in long-term aggregate persistence. Conversely, studies that measure aggregate stability at a single late time point miss the early EPS-dominated phase entirely. The model predicts that the relative importance of bacterial versus fungal binding is not a static property but evolves predictably throughout the aggregate life cycle.

\textbf{Testable predictions:}
\begin{itemize}
\item In time-series incubation experiments, peak aggregate stability should occur 6-18 months after peak CO$_2$ flux, with the lag duration increasing for larger initial POM sizes.
\item Sequential measurements of EPS content (e.g., via hot-water extraction) and fungal biomarkers (e.g., ergosterol, PLFA 18:2ω6,9) within size-separated aggregates should reveal early bacterial dominance transitioning to late fungal dominance, with a crossover around 12 months for moderate-sized POM inputs.
\item Selective fungicide or bactericide amendments should differentially affect aggregate stability depending on timing: bactericides applied during the first 6 months should suppress early aggregation, while fungicides should have minimal effect; the reverse should hold after 18 months.
\item Hyphal length density (measured via direct microscopy or qPCR of fungal ITS) within aggregates should increase continuously even as aggregate stability begins to decline, confirming that fungal abundance is decoupled from aggregate persistence once binding agents begin to degrade.
\end{itemize}

\subsection{Size-Dependent Life Cycles}

The population modeling framework treats soil as an ensemble of coevolving domains nucleated around POM inputs of diverse sizes. Individual cohorts — defined by the size and timing of their initiating POM — exhibit characteristic life cycles that differ in duration, peak stability, and dominant binding mechanisms. These size-dependent trajectories, when superimposed, generate emergent aggregate size distributions that evolve dynamically over time. Current experimental approaches measure aggregate size distributions at discrete time points but cannot directly track individual cohorts or resolve how initial POM size determines aggregate fate.

\begin{figure}
% PLACEHOLDER: Two-panel figure
% Panel A: Life cycles of individual cohorts
%   - X-axis: Time (months, 0-60)
%   - Y-axis: Aggregate diameter (mm)
%   - Four curves, one for each initial POM size: 0.1, 0.3, 0.5, 0.9 mm
%   - Environmental conditions: ψ=-33 kPa, [O2]=21%
%   - Show: Smaller POM → faster growth, earlier peak, shorter lifespan
%   - 0.1 mm POM: peaks at ~6 months at 0.3 mm diameter, vanishes by 18 months
%   - 0.3 mm POM: peaks at ~12 months at 0.7 mm diameter, vanishes by 36 months
%   - 0.5 mm POM: peaks at ~18 months at 1.2 mm diameter, vanishes by 48 months
%   - 0.9 mm POM: peaks at ~30 months at 2.0 mm diameter, persists >60 months
%   - Annotate relative growth rates and peak timing differences
% Panel B: Emergent size distribution at steady state
%   - X-axis: Aggregate diameter (mm, 0-3 mm, binned as <0.25, 0.25-0.5, 0.5-1, 1-2, >2)
%   - Y-axis: Volumetric proportion (% of total soil)
%   - Bar chart or histogram showing steady-state distribution
%   - Scenario: continuous annual POM inputs with log-normal size distribution
%   - After 100 years of annual inputs (steady state reached)
%   - Show bimodal or broad distribution with peak around 0.5-1 mm
%   - Small aggregates (<0.25 mm) represent recently nucleated cohorts
%   - Large aggregates (>2 mm) represent mature cohorts from larger POM
\caption{Size-dependent aggregate life cycles and emergent population distributions. (a) Individual cohorts initiated by POM of different sizes (0.1, 0.3, 0.5, 0.9 mm diameter) exhibit characteristic trajectories. Smaller POM particles drive faster aggregation, earlier peaks, and shorter lifespans due to more rapid substrate depletion. Larger POM particles sustain aggregation for multiple years. (b) Emergent aggregate size distribution at steady state under continuous annual POM inputs with a log-normal size distribution (mean 0.4 mm, standard deviation 0.2 mm). The steady-state distribution reflects the superposition of cohorts at different life stages, with small aggregates representing recent nucleation and large aggregates representing mature domains. The distribution is dynamic: individual aggregates continuously form, grow, and disappear, while the population-level distribution remains stable.}
\label{fig:size_dependent}
\end{figure}

The model predicts systematic size-dependent life cycle characteristics (Figure~\ref{fig:size_dependent}a). Smaller POM particles (0.1-0.3 mm) exhibit rapid aggregation, reaching peak stability within 6-12 months, but their aggregates are short-lived, disappearing within 18-36 months. Three factors contribute to this rapid cycling. First, the total carbon available for microbial activity scales with POM volume (third power of diameter), so small POM particles are quickly depleted. Second, the high surface area to volume ratio of small POM leads to rapid dissolution rates, concentrating biogeochemical activity in time. Third, small POM particles generate steeper spatial gradients in dissolved carbon due to three-dimensional dilution, producing highly localized microbial activity that exhausts substrate rapidly.

Conversely, larger POM particles (0.5-0.9 mm) support slower but more sustained aggregation. Peak stability occurs 18-36 months after nucleation, and aggregates can persist for 4-6 years or longer. The lower specific surface area reduces dissolution rates, spreading biogeochemical activity over a longer timeframe. Additionally, larger POM particles sustain higher absolute rates of fungal colonization: the volume available for hyphal growth is larger, and fungi are better able to exploit spatially extensive substrates through internal resource translocation. As a result, larger aggregates exhibit greater relative contributions from fungal binding, consistent with empirical observations that macroaggregates (>2 mm) contain higher fungal:bacterial biomass ratios than microaggregates.

When continuous POM inputs with a realistic size distribution are simulated over century timescales, an emergent steady-state aggregate size

\subsection{Age Structure and System Memory}

The age structure of aggregate populations — the distribution of cohort ages within each size class — is a fundamental but unmeasured property of soil systems. Conventional aggregation indices (mean weight diameter, proportion of water-stable aggregates) provide snapshots of current state but contain no information about how long individual aggregates have existed or how recently they formed. Yet age structure determines system vulnerability: soils dominated by young aggregates depend on continuous fresh inputs and are sensitive to disturbance, while soils with broader age distributions contain a reservoir of older, more stable aggregates that buffer against perturbation. The model explicitly tracks cohort ages through the internal coordinate $a = t - t_i$, where $t_i$ is the time of POM input that nucleated cohort $i$. This enables calculation of the age distribution $\mathcal{A}(a,d,t)$ — the number frequency of aggregates of size $d$ and age $a$ at time $t$ — and its derived properties such as mean age, age variance, and age-conditioned size distributions.

\begin{figure}
% PLACEHOLDER: Two-panel figure
% Panel A: Age distribution at steady state
%   - X-axis: Aggregate age (years, 0-15)
%   - Y-axis: Volumetric proportion (% of total aggregated soil volume)
%   - Histogram or area plot showing age distribution for one representative site
%   - Scenario: steady state after 100 years of continuous annual POM inputs
%   - Use UK site parameters (intermediate reactivity)
%   - Show: Strong skew toward young aggregates
%   - Median age ~2-3 years (mark with vertical line)
%   - >70% of aggregates <5 years old
%   - >90% of aggregates <10 years old
%   - Long tail: small proportion of aggregates 10-15 years old
% Panel B: Mean age vs. size class
%   - X-axis: Aggregate size class (<0.25, 0.25-0.5, 0.5-1, 1-2, >2 mm)
%   - Y-axis: Mean age (years)
%   - Bar chart showing mean age for each size class
%   - Error bars showing ±1 standard deviation (age variance within class)
%   - Show: Positive correlation between size and mean age
%   - <0.25 mm: mean age ~0.5 years (recently nucleated or disaggregated)
%   - 0.25-0.5 mm: mean age ~1.5 years
%   - 0.5-1 mm: mean age ~2.5 years
%   - 1-2 mm: mean age ~4 years
%   - >2 mm: mean age ~6 years (oldest, most stable)
\caption{Age structure and system memory in aggregate populations at steady state. (a) Volumetric age distribution for a representative site (UK parameters) after 100 years of continuous annual POM inputs. Despite century-scale equilibration, the aggregate population is dominated by young cohorts: median age is 2-3 years, and >90\% of aggregates are less than 10 years old. This reflects the dynamic equilibrium between continuous formation and turnover. (b) Mean age increases with aggregate size class, reflecting the longer life cycles of aggregates nucleated by larger POM. Small aggregates (<0.25 mm) include both recently formed cohorts and fragments from disaggregating larger aggregates, producing a bimodal age distribution (not shown) with mean age around 0.5 years. Large aggregates (>2 mm) are predominantly 4-8 years old, representing mature domains approaching the end of their life cycles.}
\label{fig:age_structure}
\end{figure}

At steady state, aggregate populations exhibit strongly age-skewed distributions (Figure~\ref{fig:age_structure}a). For the UK site — representing intermediate reactivity and POM input rates — the median aggregate age is approximately 2-3 years, and more than 90\% of aggregates are less than 10 years old. This result is striking: even in soils that have been undisturbed for a century or more, the vast majority of aggregates are young, having formed within the past decade. This reflects the dynamic nature of steady state: it is not a static configuration but a continuous flux of formation and disappearance, with aggregate populations maintained by the balance between annual POM inputs and cohort turnover.

The age structure varies systematically across size classes (Figure~\ref{fig:age_structure}b). Small aggregates (<0.25 mm) have mean ages around 0.5 years, representing two populations: recently nucleated cohorts from small POM inputs, and fragments produced by disaggregation of larger, older aggregates. Intermediate aggregates (0.5-1 mm) have mean ages around 2-3 years, representing cohorts in their active growth phase. Large aggregates (>2 mm) have mean ages around 4-6 years, representing mature cohorts approaching peak stability or beginning to decline. This size-age correlation is a direct consequence of the size-dependent life cycles described in Section 4.3: larger aggregates originate from larger POM and take longer to reach maturity.

The age structure quantifies system memory: current aggregate populations depend on the full history of POM inputs and environmental conditions over the past decade. A soil's current aggregation state reflects not only present conditions but also the legacy of inputs from 5-10 years prior. This has implications for interpreting management interventions: changes in land use or cropping systems will not produce their full effect on aggregate stability for several years, as pre-existing cohorts initiated under prior conditions continue to dominate the population until they turn over.

Comparison across the three field sites reveals systematic differences in age structure related to turnover rates (not shown; see Appendix~\ref{app:age_comparison}). The Australian site, with the highest reactivity, maintains the youngest population (median age 1.5 years), while the US site maintains the oldest (median age 3.5 years). This explains the differential resilience to disturbance observed in Figure~\ref{fig:field_restoration}b: young populations depend more heavily on continuous inputs and collapse more rapidly when inputs cease.

\textbf{Testable predictions:}
\begin{itemize}
\item Radiocarbon dating of size-separated aggregates should reveal the predicted size-age correlation: large aggregates (>2 mm) should have mean $^{14}$C ages 3-5 years older than small aggregates (<0.5 mm) in the same soil.
\item In soils with known land-use history (e.g., tillage cessation at a documented date), the aggregate age distribution should reflect the time since disturbance: recently restored soils should be dominated by aggregates younger than the restoration age, while century-old undisturbed soils should contain predominantly young aggregates (<10 years) due to continuous turnover.
\item Sequential $^{13}$C pulse-labeling experiments, with labeling events separated by 1-2 years, should produce cohorts with distinct isotopic signatures. Tracking these cohorts through time via size-fractionated isotopic analysis should reveal the predicted life cycle durations and turnover dynamics.
\item Aggregate age distributions should differ predictably between high-input and low-input systems: high-input systems (e.g., fertilized grasslands) should maintain younger, more rapidly cycling populations, while low-input systems (e.g., extensive rangelands) should accumulate older aggregates with broader age distributions.
\item Disturbance experiments (e.g., simulated drought, temporary cessation of vegetation) should cause predictable shifts in age structure: young cohorts should disappear first, temporarily increasing mean age, followed by gradual recovery as new cohorts form. Monitoring size-separated aggregate populations through such perturbations should reveal the predicted age-dependent vulnerability.
\end{itemize}

\subsection{Temperature-Dependent Mechanisms and Emergent Predictions}

The temperature framework introduced in Section~\ref{sec:temperature} generates several additional predictions that interact with the aggregate life-cycle dynamics described above.

\paragraph{Temperature-dependent MAOC hysteresis.}
The inequality $\mathcal{E}_{a,d} > \mathcal{E}_{a,s}$ predicts that warming narrows the hysteresis between MAOC sorption and desorption (Eq.~\ref{eq:hysteresis_ratio}). This mechanism of MAOC destabilization is distinct from, and additive with, enhanced microbial mineralization: warming both destabilizes existing MAOC through reduced kinetic trapping at mineral surfaces and accelerates its microbial consumption, potentially creating a positive feedback between soil warming and carbon loss from mineral-associated pools. This prediction is testable via paired incubation experiments at different temperatures with wetting--drying cycles, measuring MAOC accumulation and release as a function of temperature history.

\paragraph{Compensating effects on anoxic zones.}
Warming simultaneously increases O$_2$ diffusion coefficients (faster transport into the aggregate interior) and decreases O$_2$ solubility (higher $K_H$, shifting oxygen toward the gas phase). The net effect on anoxic zone extent within aggregates is model-dependent and warrants systematic sensitivity analysis. If the diffusion enhancement dominates, warming could shrink anoxic zones and reduce the carbon protection attributed to oxygen limitation; if solubility reduction dominates, the aqueous oxygen supply near the POM surface could decline despite faster gas-phase transport.

\paragraph{Differential microbial response to warming.}
The assignment $\mathcal{E}_{a,B} > \mathcal{E}_{a,F}$ (60 versus 55\,kJ\,mol$^{-1}$) implies that bacterial metabolism responds more steeply to warming than fungal metabolism. Over the aggregate life cycle, this differential sensitivity could shift the relative contributions of EPS-driven versus hyphal binding: warming would amplify the early bacterial peak while leaving the later fungal phase comparatively less affected. At the population scale, this predicts a shift in community composition within aggregates at elevated temperatures---another testable prediction amenable to PLFA or amplicon-based profiling of size-separated aggregates incubated at different temperatures.

\paragraph{POM dissolution coupled to microbial colonization.}
The revised POM dissolution rate (Section~2.4.1) now depends explicitly on bacterial and fungal biomass at the POM surface through dual Monod terms. This creates a positive feedback absent in the previous formulation: more microbes produce more extracellular enzymes, accelerating dissolution, which releases more DOC, supporting further microbial growth. The dual half-saturation form ($K_{B,P}$, $K_{F,P}$) prevents runaway feedback by saturating the enzymatic contribution at high biomass. This coupling predicts that sterile or sparsely colonized POM should dissolve substantially more slowly than POM in microbially active soil---a prediction testable by comparing dissolution rates in sterilized versus inoculated incubations.


\conclusions  %% \conclusions[modified heading if necessary]
TEXT

%% The following commands are for the statements about the availability of data sets and/or software code corresponding to the manuscript.
%% It is strongly recommended to make use of these sections in case data sets and/or software code have been part of your research the article is based on.

\codeavailability{TEXT} %% use this section when having only software code available


\dataavailability{TEXT} %% use this section when having only data sets available


\codedataavailability{TEXT} %% use this section when having data sets and software code available


\sampleavailability{TEXT} %% use this section when having geoscientific samples available


\videosupplement{TEXT} %% use this section when having video supplements available


\appendix
\section*{Supplemental Materials}    %% Appendix A
\appendix

\section{Aggregate Size Distribution Evolution in Laboratory Incubations}

The model predicts the temporal evolution of aggregate size distributions for both incubation experiments, extending beyond the experimental timespan where data are available.

\begin{figure}
% PLACEHOLDER: Multi-panel figure for wheat straw experiment
% Six panels arranged in 2 rows × 3 columns
% Each panel shows aggregate size distribution (bar chart or histogram)
%   - X-axis: Aggregate size classes (<0.25, 0.25-0.5, 0.5-1, 1-2, >2 mm)
%   - Y-axis: Proportion of total soil volume (%)
% Panels show distributions at: 1 week, 2 weeks, 3 weeks, 1 month, 1.5 months, 2 months
% First three panels are within experimental timespan; last three are extrapolations
\caption{Predicted temporal evolution of aggregate size distribution for the wheat straw incubation experiment. Distributions are shown at 1, 2, and 3 weeks (within experimental timespan) and at 1, 1.5, and 2 months (model extrapolations). During the first three weeks, intermediate aggregates (0.5-2 mm) emerge rapidly. By the second month, large aggregates ($>2$ mm) dominate, accounting for approximately 60\% of total soil volume.}
\label{fig:size_dist_wheat}
\end{figure}

\begin{figure}
% PLACEHOLDER: Multi-panel figure for manure experiment
% Six panels arranged in 2 rows × 3 columns
% Each panel shows aggregate size distribution
%   - X-axis: Aggregate size classes (<0.25, 0.25-0.5, 0.5-1, 1-2, >2 mm)
%   - Y-axis: Proportion of total soil volume (%)
% Panels show distributions at: 3, 6, 12, 18, 24, 36 months
% First four panels are within experimental timespan; last two are extrapolations
\caption{Predicted temporal evolution of aggregate size distribution for the manure incubation experiment. Distributions are shown at 3, 6, 12, and 18 months (within experimental timespan) and at 24 and 36 months (model extrapolations). Large aggregates (1-2 mm and $>2$ mm) increase in proportion through 12 months, reaching a peak, then decline as binding agents degrade. Fine aggregates ($<0.25$ mm) increase after 18 months as larger aggregates disaggregate, maintaining a small but relatively constant fraction.}
\label{fig:size_dist_manure}
\end{figure}

For the wheat straw experiment (Figure~\ref{fig:size_dist_wheat}), aggregates in the 0.5-2 mm range emerged within one week. By the third week, large aggregates (1-2 mm and $>2$ mm) accounted for approximately 60\% of total soil volume. Model extrapolations predict continued growth of the $>2$ mm fraction through the second month, followed by stabilization as POM depletion limits further binding agent production.

For the manure experiment (Figure~\ref{fig:size_dist_manure}), the proportion of large aggregates increased progressively through 12 months, consistent with the slower decomposition of manure relative to wheat straw. After 18 months, disaggregation of intermediate-sized aggregates produces an increasing fraction of fine aggregates ($<0.25$ mm and 0.25-0.5 mm), which stabilize at approximately 15-20\% of total volume. This redistribution reflects the natural life cycle of aggregates: formation around POM, growth through binding agent accumulation, and eventual breakdown as resources are exhausted.

\section{Steady-State Pool Distributions and Seasonal Dynamics}

\begin{figure}
% PLACEHOLDER: Three-panel bar chart
% One panel for each field site (US, UK, Australia)
% Each panel shows carbon pool distribution at steady state (year 15)
%   - X-axis: Carbon pools (POM, DOC, MAOC, Biomass [B+F], EPS)
%   - Y-axis: Carbon content (g C / kg soil)
%   - Stacked or grouped bars showing relative contributions
% Australia site has highest total C and largest MAOC fraction
% US site has lowest total C but relatively high MAOC:POM ratio (older system)
\caption{Model-predicted steady-state carbon pool distributions for the three field restoration sites at year 15. MAOC dominates total carbon storage at all three sites, accounting for 50-65\% of total soil organic carbon. The US site, despite having the lowest total carbon, exhibits the highest MAOC:POM ratio, reflecting its slower turnover and older mean aggregate age. These predictions are testable through standard fractionation protocols.}
\label{fig:ss_pools}
\end{figure}

\begin{figure}
% PLACEHOLDER: Three-panel time series (one for each site)
% Each panel shows 5-year window at steady state (e.g., years 95-100)
%   - X-axis: Time (monthly resolution over 5 years)
%   - Y-axis: Aggregated volume (% of total)
% Show oscillations driven by annual POM input cycles
% Australia site shows largest amplitude oscillations (high turnover)
% US site shows smallest amplitude (low turnover, buffered by old aggregates)
\caption{Model-predicted seasonal fluctuations in aggregate volume at steady state for the three field sites. All three sites exhibit oscillations driven by annual cycles of POM input (e.g., root turnover in spring/summer). The amplitude of oscillations correlates with site reactivity: the Australian site, with rapid turnover, shows fluctuations of $\pm 8$\% around the mean, while the US site shows fluctuations of only $\pm 2$\%. These patterns reflect the aggregate age structure — young populations respond more strongly to pulsed inputs. Such seasonal fluctuations have been observed in field studies \citep{Low1955,Blackman1992,Coote1988} but are not typically captured in empirical models.}
\label{fig:seasonal}
\end{figure}

At steady state, MAOC dominates total carbon storage at all three sites (Figure~\ref{fig:ss_pools}), accounting for 50-65\% of soil organic carbon. The US site, despite having the lowest total carbon content, exhibits the highest MAOC:POM ratio (approximately 3:1), reflecting its slower turnover dynamics and older mean aggregate age. In contrast, the Australian site maintains a larger fraction of carbon in active POM and microbial biomass pools due to continuous rapid cycling.

The model predicts seasonal fluctuations in aggregate volume at steady state, driven by annual cycles of POM input (Figure~\ref{fig:seasonal}). The amplitude of these oscillations correlates with site reactivity: systems with rapid turnover exhibit larger fluctuations, while systems with slower turnover are buffered by the presence of older, more stable aggregates. These predicted seasonal dynamics are consistent with field observations \citep{Low1955,Blackman1992,Coote1988} but are not typically captured by empirical aggregation indices or static carbon pool models.

\section{Biophysical Controls on Domain Formation}

The model enables systematic exploration of how environmental and soil properties influence aggregate formation dynamics. We evaluated four key factors — water content (matric potential), ambient oxygen concentration, soil texture (median particle size), and initial POM size — by varying each across a wide range while holding the others constant. These sensitivity analyses demonstrate the model's capacity to reproduce observed empirical relationships and generate mechanistic explanations for parameter dependencies.

\begin{figure}
% PLACEHOLDER: Four-panel figure
% Panel A: Water content (matric potential) effects
%   - X-axis: Time (months, 0-60)
%   - Y-axis: Aggregate diameter (mm)
%   - Multiple curves for different ψ values: -1, -10, -30, -100, -300, -1000, -2000 kPa
%   - Initial POM: 0.5 mm, [O2]=21%, median particle size=50 μm
%   - Show: Wetter conditions (ψ > -30 kPa) → faster aggregation, earlier peak, shorter lifespan
%   - Drier conditions (ψ < -300 kPa) → slower aggregation, delayed peak, longer lifespan
%   - Very dry (ψ=-2000 kPa) → delayed onset (6 months before aggregation begins)
% Panel B: Oxygen concentration effects
%   - X-axis: Time (months, 0-60)
%   - Y-axis: Aggregate diameter (mm)
%   - Multiple curves for different [O2]: 21%, 15%, 10%, 5%, 2%
%   - Initial POM: 0.5 mm, ψ=-33 kPa, median particle size=50 μm
%   - Show: Reduced [O2] → smaller peak aggregate size
%   - Time to peak remains similar across [O2] levels
%   - At [O2]=2%, peak aggregate diameter ~50% of aerobic value
% Panel C: Soil texture (particle size) effects
%   - X-axis: Time (months, 0-60)
%   - Y-axis: Aggregate diameter (mm)
%   - Multiple curves for different median particle sizes: 5, 10, 25, 50, 100 μm
%   - Initial POM: 0.5 mm, ψ=-33 kPa, [O2]=21%
%   - Show: Finer textures → much larger aggregates (5× difference between 5 and 100 μm)
%   - Early EPS-driven aggregation much more pronounced in fine soils
%   - Coarser soils require stronger fungal binding, producing smaller, later peaks
% Panel D: Initial POM size effects
%   - X-axis: Time (months, 0-60)
%   - Y-axis: Aggregate diameter normalized by initial POM diameter (dimensionless)
%   - Multiple curves for initial POM diameter: 0.1, 0.3, 0.5, 0.9, 1.5 mm
%   - ψ=-33 kPa, [O2]=21%, median particle size=50 μm
%   - Show: Smaller POM → faster relative growth, higher normalized peak, shorter lifespan
%   - 0.1 mm POM: normalized peak ~3×, lifespan <12 months
%   - 1.5 mm POM: normalized peak ~1.5×, lifespan >60 months
\caption{Biophysical controls on aggregate formation. (a) Water content: aggregates formed under humid conditions (ψ ≥ -30 kPa) reach peak size rapidly but disappear within 5 years. Aggregates formed under dry conditions (ψ ≤ -300 kPa) grow slowly, with delayed onset, but persist longer. (b) Oxygen concentration: reduced [O$_2$] inhibits binding agent production more than POM dissolution, reducing peak aggregate size without altering life cycle duration. (c) Soil texture: finer soils produce much larger aggregates due to lower particle detachment forces. EPS is more effective in fine soils; coarse soils rely more heavily on fungal binding. (d) Initial POM size: smaller POM produces faster relative growth and higher normalized peak diameter but shorter absolute lifespan. Larger POM sustains aggregation for years but grows more slowly relative to its initial size.}
\label{fig:biophysical_controls}
\end{figure}

\subsection{Water Content}

The model predicts that aggregate turnover rate (life cycle duration) is positively correlated with soil wetness (Figure~\ref{fig:biophysical_controls}a). Matric potential was varied from drier than permanent wilting point to near saturation (-2000 kPa ≤ ψ ≤ -1 kPa). Aggregates formed under humid conditions (ψ ≥ -30 kPa) reached peak size in less than 12 months and disappeared by 60 months. In contrast, aggregates formed under very dry conditions (ψ = -1000 to -2000 kPa) grew slowly, reaching peak size 36-48 months after POM input, and persisted beyond the 60-month simulation window.

This moisture dependence reflects the exponential sensitivity of microbial uptake rates to water potential (the $e^{\nu\psi}$ terms in $R_B$ and $R_F$). Under dry conditions, microbial activity is suppressed, slowing both POM decomposition and binding agent production. The time required to accumulate sufficient EPS and fungal biomass at the POM surface is延长ed, producing delayed onset of aggregation (up to 6 months at ψ = -2000 kPa). However, once aggregation begins, the slower pace of carbon cycling extends the period over which binding agents remain above the stability threshold, prolonging aggregate persistence.

\subsection{Oxygen Availability}

Reduced ambient oxygen concentration inhibits aggregation without altering life cycle duration (Figure~\ref{fig:biophysical_controls}b). Oxygen was varied from atmospheric (21\%) to severely hypoxic (2\%). Peak aggregate diameter declined progressively with decreasing [O$_2$], reaching approximately 50\% of the aerobic value at [O$_2$] = 2\%. However, the time to peak aggregation remained nearly constant across all oxygen levels.

This pattern reflects the differential sensitivity of POM dissolution versus binding agent production to oxygen limitation. POM dissolution (Eq. in Section~2.4.1) depends on oxygen through a Monod term with half-saturation constant $L_P$. Microbial uptake and growth (Eqs. in Section~2.4.3) depend on oxygen through similar Monod terms with half-saturation constants $L_B$ and $L_F$. The model parameterization assumes $L_P < L_B, L_F$, meaning POM dissolution is less sensitive to oxygen limitation than microbial metabolism. Consequently, under hypoxia, POM continues to dissolve at near-normal rates, but microbes grow more slowly and produce fewer binding agents. The overall carbon turnover rate — and thus aggregate life cycle duration — is determined primarily by POM dissolution, explaining why reduced oxygen affects aggregate size but not timing.

\subsection{Soil Texture}

Soil texture exerts a strong control on aggregate size through its effect on particle detachment forces (Figure~\ref{fig:biophysical_controls}c). Median particle diameter was varied from 5 μm (clay-sized) to 100 μm (fine sand). Given similar distributions of binding agents (EPS and hyphae), aggregates in soils with median grain size of 5 μm are more than 5 times larger than aggregates in soils with median grain size of 100 μm.

This texture dependence arises from the stability criterion (Section~2.5). The drag stress during wet sieving is proportional to $1/r_{\text{eff}}$, where $r_{\text{eff}} = r + \chi a_p$ includes the roughness contribution from primary particles. Finer soils have smaller $a_p$, reducing $r_{\text{eff}}$ and thus increasing the drag stress that must be overcome. However, the cohesive stress from binding agents is independent of particle size. Therefore, in fine-textured soils, small amounts of EPS can generate sufficient binding to withstand sieving, producing large aggregates early in the life cycle. In coarse-textured soils, stronger binding from fungal hyphae is required, producing smaller aggregates that emerge later.

The early influence of EPS-driven aggregation is much more pronounced in finer soils, while coarser soils exhibit greater relative contributions from fungal binding. This texture-dependent shift in binding agent importance is consistent with empirical observations of positive correlations between aggregate stability and clay plus silt content \citep{Degens1996,Idowu2003,Skidmore2010}.

\subsection{Initial POM Size}

Initial POM size influences aggregation through three interacting factors (Figure~\ref{fig:biophysical_controls}d). First, total available carbon scales with the cube of POM diameter. Second, spatial dilution of dissolved carbon scales with the square of distance from the POM surface. Third, specific surface area for dissolution scales inversely with diameter. The combination produces rapid, high-concentration carbon pulses around small POM and slower, sustained release from large POM.

When normalized by initial POM diameter, small POM particles (0.1-0.3 mm) produce much larger relative growth (normalized peak diameter 2.5-3×) and shorter lifespans (<18 months). Large POM particles (0.9-1.5 mm) produce smaller relative growth (normalized peak 1.5-2×) but sustain aggregation for 4-6 years or longer. The rapid cycling of small POM produces aggregates dominated by bacterial EPS, while large POM sustains significant fungal colonization and hyphal binding.

These predictions are consistent with empirical observations that fine organic matter particles produce rapid aggregation responses in short-term incubations, while coarse woody debris or large root fragments sustain aggregation over multi-year timescales.

\section{Radial Profiles of Biogeochemical State Variables}

\begin{figure}
% PLACEHOLDER: Four-panel figure showing radial profiles at selected times
% Each panel shows spatial distribution vs. radial distance (r, mm) from POM core
% Four panels: (a) Dissolved carbon C, (b) Oxygen O, (c) Bacterial biomass B, (d) Fungal biomass F_i
% Each panel contains 4-5 curves representing different time points: 1, 6, 12, 24, 48 months
% Scenario: 0.5 mm initial POM, ψ=-33 kPa, [O2]=21%
% X-axis for all: Radial distance r (mm, 0-5 mm)
% Y-axis varies by panel
% Panel A (C): 
%   - Sharp peak near r=r_0 (POM surface) at early times
%   - Peak declines and broadens over time as POM depletes
%   - By 48 months, nearly flat profile at low concentration
% Panel B (O):
%   - Depletion near POM surface (anoxic or hypoxic core)
%   - Gradient steepest at 6-12 months when microbial activity is highest
%   - Recovers toward ambient [O2] at large r and late times
% Panel C (B):
%   - Highly localized near POM surface
%   - Peaks at 6-12 months, then declines
%   - Nearly zero beyond 1-2 mm from POM at all times
% Panel D (F_i):
%   - More spatially extensive than bacteria
%   - Increases throughout domain over first 24 months
%   - Persists at significant concentrations to 3-4 mm even at late times
\caption{Radial profiles of biogeochemical state variables at selected times. (a) Dissolved organic carbon exhibits a sharp peak near the POM surface at early times, declining as the substrate is exhausted. (b) Oxygen is depleted near the POM surface, creating hypoxic or transiently anoxic conditions during peak microbial activity (6-12 months). (c) Bacterial biomass is highly localized within 1-2 mm of the POM surface, reflecting rapid consumption of dissolved carbon and limited motility. (d) Fungal biomass ($F_i$, insulated hyphae) extends 3-4 mm from the POM surface due to internal resource translocation, creating spatially extensive binding networks that persist long after bacteria have declined. These profiles demonstrate the spatial organization predicted by the POM-centric mechanism but not directly observable with current experimental methods.}
\label{fig:radial_profiles}
\end{figure}

The radial profiles reveal the spatial structure of biogeochemical hotspots (Figure~\ref{fig:radial_profiles}). Dissolved organic carbon is highly concentrated near the POM surface, declining sharply with distance due to microbial uptake, equilibrium sorption, and geometric dilution. Bacterial biomass is similarly localized, remaining within 1-2 mm of the POM throughout the life cycle. This localization reflects bacteria's dependence on dissolved substrates and their limited motility (diffusivity $D_B \approx 0.5 D_C$).

Oxygen exhibits depletion near the POM surface, with gradients steepest during the period of peak microbial activity (6-12 months). At this stage, dissolved oxygen concentration at the POM surface can drop to 10-20\% of ambient levels, approaching hypoxic or transiently anoxic conditions depending on POM size and moisture content. This provides a mechanistic basis for testing the anoxic core hypothesis invoked in conceptual models of aggregate-mediated carbon protection.

Fungal biomass exhibits fundamentally different spatial organization. Internal resource translocation via the mobile pool ($F_m$) allows fungi to extend 3-4 mm from the POM surface, well beyond the zone of elevated dissolved carbon. Insulated hyphae ($F_i$) — the primary binding agents — accumulate progressively throughout this volume, creating spatially extensive networks that persist for years after bacterial populations have collapsed. This spatial extent explains fungi's disproportionate contribution to aggregate stability relative to their biomass: hyphal networks physically enmesh mineral particles across centimeter scales, while bacterial EPS acts primarily at the sub-millimeter scale near the POM core.

\section{Comparison of Steady-State Age Distributions Across Field Sites}
\label{app:age_comparison}

\begin{figure}
% PLACEHOLDER: Three-panel figure
% One panel for each field site (US, UK, Australia)
% Each panel shows volumetric age distribution at steady state (year 100)
%   - X-axis: Aggregate age (years, 0-15)
%   - Y-axis: Volumetric proportion (% of total aggregated soil)
%   - Histogram or area plot
% US site (slowest reactivity):
%   - Broader distribution, median age ~3.5 years
%   - Significant tail extending to 12-15 years
%   - ~75% < 10 years old
% UK site (intermediate):
%   - Median age ~2.5 years
%   - ~85% < 10 years old
% Australia site (fastest reactivity):
%   - Narrow distribution, median age ~1.5 years
%   - >95% < 10 years old, almost none >12 years
%   - Sharp cutoff reflects rapid turnover
\caption{Steady-state age distributions for the three field restoration sites. The US site (slowest reactivity, lowest POM input rate) maintains the oldest population with median age 3.5 years. The Australia site (highest reactivity, highest POM input rate) maintains the youngest population with median age 1.5 years. Despite century-scale equilibration, all three sites are dominated by aggregates less than 10 years old, confirming that steady state is a dynamic equilibrium of continuous formation and turnover rather than a static configuration.}
\label{fig:age_comparison}
\end{figure}

The age distributions at steady state differ systematically across the three field sites (Figure~\ref{fig:age_comparison}), reflecting their contrasting POM input rates and reactivity. The US site, with the lowest reactivity (0.25×) and lowest POM input rate (0.48\% yr$^{-1}$), maintains the oldest aggregate population. The median age is approximately 3.5 years, and a substantial fraction (25\%) of aggregates are older than 10 years. This older population provides a reservoir of stable aggregates that buffer the system against disturbance, explaining the gradual decline observed in Figure~\ref{fig:field_restoration}b when POM inputs are reduced.

The Australia site, with the highest reactivity (1.0×) and highest POM input rate (6.8\% yr$^{-1}$), maintains the youngest population. The median age is only 1.5 years, and more than 95\% of aggregates are less than 10 years old. This rapid turnover makes the system highly responsive to changes in POM inputs — aggregates form quickly when substrate is available but disappear rapidly when inputs cease. This explains the steep collapse observed under simulated disturbance.

The UK site occupies an intermediate position, with median age around 2.5 years. These age structure differences quantify a fundamental trade-off: fast-cycling systems maximize short-term carbon processing and nutrient turnover but are vulnerable to disturbance, while slow-cycling systems build resilience through accumulation of older aggregates but respond sluggishly to management interventions.

\section*{LAST}

\noappendix       %% use this to mark the end of the appendix section. Otherwise the figures might be numbered incorrectly (e.g. 10 instead of 1).

%% Regarding figures and tables in appendices, the following two options are possible depending on your general handling of figures and tables in the manuscript environment:

%% Option 1: If you sorted all figures and tables into the sections of the text, please also sort the appendix figures and appendix tables into the respective appendix sections.
%% They will be correctly named automatically.

%% Option 2: If you put all figures after the reference list, please insert appendix tables and figures after the normal tables and figures.
%% To rename them correctly to A1, A2, etc., please add the following commands in front of them:

\appendixfigures  %% needs to be added in front of appendix figures

\appendixtables   %% needs to be added in front of appendix tables

%% Please add \clearpage between each table and/or figure. Further guidelines on figures and tables can be found below.



\authorcontribution{TEXT} %% this section is mandatory

\competinginterests{TEXT} %% this section is mandatory even if you declare that no competing interests are present

\disclaimer{TEXT} %% optional section

\begin{acknowledgements}
TEXT
\end{acknowledgements}




%% REFERENCES

%% The reference list is compiled as follows:

%\begin{thebibliography}{}
%
%\bibitem[AUTHOR(YEAR)]{LABEL1}
%REFERENCE 1
%
%\bibitem[AUTHOR(YEAR)]{LABEL2}
%REFERENCE 2
%
%\end{thebibliography}

%% Since the Copernicus LaTeX package includes the BibTeX style file copernicus.bst,
%% authors experienced with BibTeX only have to include the following two lines:
%%
\bibliographystyle{copernicus}
\bibliography{references,refs_temp}
%%
%% URLs and DOIs can be entered in your BibTeX file as:
%%
%% URL = {http://www.xyz.org/~jones/idx_g.htm}
%% DOI = {10.5194/xyz}


%% LITERATURE CITATIONS
%%
%% command                        & example result
%% \citet{jones90}|               & Jones et al. (1990)
%% \citep{jones90}|               & (Jones et al., 1990)
%% \citep{jones90,jones93}|       & (Jones et al., 1990, 1993)
%% \citep[p.~32]{jones90}|        & (Jones et al., 1990, p.~32)
%% \citep[e.g.,][]{jones90}|      & (e.g., Jones et al., 1990)
%% \citep[e.g.,][p.~32]{jones90}| & (e.g., Jones et al., 1990, p.~32)
%% \citeauthor{jones90}|          & Jones et al.
%% \citeyear{jones90}|            & 1990



%% FIGURES

%% When figures and tables are placed at the end of the MS (article in one-column style), please add \clearpage
%% between bibliography and first table and/or figure as well as between each table and/or figure.

% The figure files should be labelled correctly with Arabic numerals (e.g. fig01.jpg, fig02.png).


%% ONE-COLUMN FIGURES

%%f
%\begin{figure}[t]
%\includegraphics[width=8.3cm]{FILE NAME}
%\caption{TEXT}
%\end{figure}
%
%%% TWO-COLUMN FIGURES
%
%%f
%\begin{figure*}[t]
%\includegraphics[width=12cm]{FILE NAME}
%\caption{TEXT}
%\end{figure*}
%
%
%%% TABLES
%%%
%%% The different columns must be seperated with a & command and should
%%% end with \\ to identify the column brake.
%
%%% ONE-COLUMN TABLE
%
%%t
%\begin{table}[t]
%\caption{TEXT}
%\begin{tabular}{column = lcr}
%\tophline
%
%\middlehline
%
%\bottomhline
%\end{tabular}
%\belowtable{} % Table Footnotes
%\end{table}
%
%%% TWO-COLUMN TABLE
%
%%t
%\begin{table*}[t]
%\caption{TEXT}
%\begin{tabular}{column = lcr}
%\tophline
%
%\middlehline
%
%\bottomhline
%\end{tabular}
%\belowtable{} % Table Footnotes
%\end{table*}
%
%%% LANDSCAPE TABLE
%
%%t
%\begin{sidewaystable*}[t]
%\caption{TEXT}
%\begin{tabular}{column = lcr}
%\tophline
%
%\middlehline
%
%\bottomhline
%\end{tabular}
%\belowtable{} % Table Footnotes
%\end{sidewaystable*}
%
%
%%% MATHEMATICAL EXPRESSIONS
%
%%% All papers typeset by Copernicus Publications follow the math typesetting regulations
%%% given by the IUPAC Green Book (IUPAC: Quantities, Units and Symbols in Physical Chemistry,
%%% 2nd Edn., Blackwell Science, available at: http://old.iupac.org/publications/books/gbook/green_book_2ed.pdf, 1993).
%%%
%%% Physical quantities/variables are typeset in italic font (t for time, T for Temperature)
%%% Indices which are not defined are typeset in italic font (x, y, z, a, b, c)
%%% Items/objects which are defined are typeset in roman font (Car A, Car B)
%%% Descriptions/specifications which are defined by itself are typeset in roman font (abs, rel, ref, tot, net, ice)
%%% Abbreviations from 2 letters are typeset in roman font (RH, LAI)
%%% Vectors are identified in bold italic font using \vec{x}
%%% Matrices are identified in bold roman font
%%% Multiplication signs are typeset using the LaTeX commands \times (for vector products, grids, and exponential notations) or \cdot
%%% The character * should not be applied as mutliplication sign
%
%
%%% EQUATIONS
%
%%% Single-row equation
%
%\begin{equation}
%
%\end{equation}
%
%%% Multiline equation
%
%\begin{align}
%& 3 + 5 = 8\\
%& 3 + 5 = 8\\
%& 3 + 5 = 8
%\end{align}
%
%
%%% MATRICES
%
%\begin{matrix}
%x & y & z\\
%x & y & z\\
%x & y & z\\
%\end{matrix}
%
%
%%% ALGORITHM
%
%\begin{algorithm}
%\caption{...}
%\label{a1}
%\begin{algorithmic}
%...
%\end{algorithmic}
%\end{algorithm}
%
%
%%% CHEMICAL FORMULAS AND REACTIONS
%
%%% For formulas embedded in the text, please use \chem{}
%
%%% The reaction environment creates labels including the letter R, i.e. (R1), (R2), etc.
%
%\begin{reaction}
%%% \rightarrow should be used for normal (one-way) chemical reactions
%%% \rightleftharpoons should be used for equilibria
%%% \leftrightarrow should be used for resonance structures
%\end{reaction}
%
%
%%% PHYSICAL UNITS
%%%
%%% Please use \unit{} and apply the exponential notation


\end{document}
